%!TEX encoding = UTF-8 Unicode
%!TEX TS-program = xelatex

%\documentclass[11pt, a4paper]{article}
\documentclass[parskip, DIV=14, pagenumber=head,top]{scrartcl}
\usepackage{fontspec}
\usepackage{xltxtra} % a few fixes and extras
\usepackage{fancyhdr} % header


\usepackage{xcolor} % writing in color

\renewcommand{\baselinestretch}{1.2}% interligne + 20%

%\setmainfont[Ligatures=TeX]{Linux Libertine}
%\setmainfont[Ligatures=TeX]{Sanskrit 2003}
\setmainfont[Ligatures=TeX]{Chandas}

% footnote in end of document
\interfootnotelinepenalty=10000

\renewcommand{\thefootnote}{\fnsymbol{footnote}} %A sequence of nine symbols, try it and see!

% add a small title: name of the file + initials
\pagestyle{fancy}
\lhead{\color{lightgray} candrāṣṭhottaraśatanāmāvalī}
\rhead{\color{lightgray} LSN}

\setkomafont{pagenumber}{\normalfont\bfseries}
\pagenumbering{roman}


\begin{document}

%\pagenumbering{gobble} % remove page numebr

\vspace{-1.5cm}

\begin{center}
% see command forfootmark without numbering
\textbf{{\Huge॥\,~candrāṣṭhottaraśatanāmāvalī\,~॥\LARGE\let\thefootnote\relax\footnote{\color{lightgray} Document written using \XeLaTeX{} and chandas font, version 03/06/2016, LSN.}}}
\end{center}
\Large

\centering	
%\raggedright

॥\,candra bīja mantra\,॥\\
\vspace{0.5cm}

oṁ śrāṁ śrīṁ śrauṁ saḥ candrāya namaḥ\,~।\\
\vspace{1.5cm}
oṁ śrīmate namaḥ\,~।\\
oṁ śaśadharāya namaḥ\,~।\\
oṁ candrāya namaḥ\,~।\\
oṁ tārādhīśāya namaḥ\,~।\\
oṁ niśākarāya namaḥ\,~।\\
oṁ sukhanidhaye namaḥ\,~।\\
oṁ sadārādhyāya namaḥ\,~।\\
oṁ satpataye namaḥ\,~।\\
oṁ sādhupūjitāya namaḥ\,~।\\
oṁ jitendriyāya namaḥ\,~॥\,10\,॥\\ 
oṁ jayodyogāya namaḥ\,~।\\
oṁ jyotiścakrapravartakāya namaḥ\,~।\\
oṁ vikartanānujāya namaḥ\,~।\\
oṁ vīrāya namaḥ\,~।\\
oṁ viśveśāya namaḥ\,~।\\
oṁ viduṣāṁ pataye namaḥ\,~।\\
oṁ doṣākarāya namaḥ\,~।\\
oṁ duṣṭadūrāya namaḥ\,~।\\
oṁ puṣṭimate namaḥ\,~।\\
oṁ śiṣṭapālakāya namaḥ\,~॥\,20\,॥\\ 
oṁ aṣṭamūrtipriyāya namaḥ\,~।\\
oṁ anantāya namaḥ\,~।\\
oṁ kaṣṭadārukuṭhārakāya namaḥ\,~।\\
oṁ svaprakāśāya namaḥ\,~।\\
oṁ prakāśātmane namaḥ\,~।\\
oṁ dyucarāya namaḥ\,~।\\
oṁ devabhojanāya namaḥ\,~।\\
oṁ kalādharāya namaḥ\,~।\\
oṁ kālahetave namaḥ\,~।\\
oṁ kāmakṛte namaḥ\,~॥\,30\,॥\\ 
oṁ kāmadāyakāya namaḥ\,~।\\
oṁ mṛtyusaṁhārakāya namaḥ\,~।\\
oṁ amartyāya namaḥ\,~।\\
oṁ nityānuṣṭhānadāyakāya namaḥ\,~।\\
oṁ kṣapākarāya namaḥ\,~।\\
oṁ kṣīṇapāpāya namaḥ\,~।\\
oṁ kṣayavṛddhisamanvitāya namaḥ\,~।\\
oṁ jaivātṛkāya namaḥ\,~।\\
oṁ śucaye namaḥ\,~।\\
oṁ śubhrāya namaḥ\,~॥\,40\,॥\\ 
oṁ jayine namaḥ\,~।\\
oṁ jayaphalapradāya namaḥ\,~।\\
oṁ sudhāmayāya namaḥ\,~।\\
oṁ surasvāmine namaḥ\,~।\\
oṁ bhaktānāmiṣṭadāyakāya namaḥ\,~।\\
oṁ bhuktidāya namaḥ\,~।\\
oṁ muktidāya namaḥ\,~।\\
oṁ bhadrāya namaḥ\,~।\\
oṁ bhaktadāridryabhañjakāya namaḥ\,~।\\
var  oṁ bhaktadāridryabhañjanāya namaḥ\,~।\\
oṁ sāmagānapriyāya namaḥ\,~॥\,50\,॥\\ 
oṁ sarvarakṣakāya namaḥ\,~।\\
oṁ sāgarodbhavāya namaḥ\,~।\\
oṁ bhayāntakṛte namaḥ\,~।\\
oṁ bhaktigamyāya namaḥ\,~।\\
oṁ bhavabandhavimocakāya namaḥ\,~।\\
oṁ jagatprakāśakiraṇāya namaḥ\,~।\\
oṁ jagadānandakāraṇāya namaḥ\,~।\\
oṁ nissapatnāya namaḥ\,~।\\
oṁ nirāhārāya namaḥ\,~।\\
oṁ nirvikārāya namaḥ\,~॥\,60\,॥\\ 
oṁ nirāmayāya namaḥ\,~।\\
oṁ bhūcchayā''cchāditāya namaḥ\,~।\\
oṁ bhavyāya namaḥ\,~।\\
oṁ bhuvanapratipālakāya namaḥ\,~।\\
oṁ sakalārtiharāya namaḥ\,~।\\
oṁ saumyajanakāya namaḥ\,~।\\
oṁ sādhuvanditāya namaḥ\,~।\\
oṁ sarvāgamajñāya namaḥ\,~।\\
oṁ sarvajñāya namaḥ\,~।\\
oṁ sanakādimunistutāya namaḥ\,~॥\,70\,॥\\ 
oṁ sitacchatradhvajopetāya namaḥ\,~।\\
oṁ sitāṅgāya namaḥ\,~।\\
oṁ sitabhūṣaṇāya namaḥ\,~।\\
oṁ śvetamālyāmbaradharāya namaḥ\,~।\\
oṁ śvetagandhānulepanāya namaḥ\,~।\\
oṁ daśāśvarathasaṁrūḍhāya namaḥ\,~।\\
oṁ daṇḍapāṇaye namaḥ\,~।\\
oṁ dhanurdharāya namaḥ\,~।\\
oṁ kundapuṣpojjvalākārāya namaḥ\,~।\\
oṁ nayanābjasamudbhavāya namaḥ\,~॥\,80\,॥\\ 
oṁ ātreyagotrajāya namaḥ\,~।\\
oṁ atyantavinayāya namaḥ\,~।\\
oṁ priyadāyakāya namaḥ\,~।\\
oṁ karuṇārasasampūrṇāya namaḥ\,~।\\
oṁ karkaṭaprabhave namaḥ\,~।\\
oṁ avyayāya namaḥ\,~।\\
oṁ caturaśrāsanārūḍhāya namaḥ\,~।\\
oṁ caturāya namaḥ\,~।\\
oṁ divyavāhanāya namaḥ\,~।\\
oṁ vivasvanmaṇḍalāgneyavāsase namaḥ\,~॥\,90\,॥\\ 
oṁ vasusamṛddhidāya namaḥ\,~।\\
oṁ maheśvarapriyāya namaḥ\,~।\\
oṁ dāntāya namaḥ\,~।\\
oṁ merugotrapradakṣiṇāya namaḥ\,~।\\
oṁ grahamaṇḍalamadhyasthāya namaḥ\,~।\\
oṁ grasitārkāya namaḥ\,~।\\
oṁ grahādhipāya namaḥ\,~।\\
oṁ dvijarājāya namaḥ\,~।\\
oṁ dyutilakāya namaḥ\,~।\\
oṁ dvibhujāya namaḥ\,~॥\,100\,॥\\ 
oṁ dvijapūjitāya namaḥ\,~।\\
oṁ audumbaranagāvāsāya namaḥ\,~।\\
oṁ udārāya namaḥ\,~।\\
oṁ rohiṇīpataye namaḥ\,~।\\
oṁ nityodayāya namaḥ\,~।\\
oṁ munistutyāya namaḥ\,~।\\
oṁ nityānandaphalapradāya namaḥ\,~।\\
oṁ sakalāhlādanakarāya namaḥ\,~॥\,108\,॥\\ 

oṁ palāśedhmapriyāya namaḥ\,~।\\
oṁ palāśasamidhapriyāya namaḥ\,~।\\

\vspace{1cm}

॥~\,iti candrāṣṭottaraśatanāmāvaliḥ sampūrṇā\,~॥ \\

%Propitiation of the Moon (Monday)
%CHARITY: Donate water, cow's milk or white rice
%to a female leader on Monday evening.
%FASTING: On Mondays, especially during Moon
%transits and major or minor Moon periods.
%MANTRA: To be chanted on Monday evening,
%RESULT: The planetary diety Chandra is propitiated
%increasing mental health and peace of mind.
%Some details at http://members.tripod.com/~navagraha
%
%Transliteration and information by
%Dr. S. Kalyanaraman kalyan97@yahoo.com. 
%Proofread by Detlef Eichler DetlefEichler@gmx.net
%Sunder Hattangadi, PSA Easwaran





\end{document}