%!TEX encoding = UTF-8 Unicode
%!TEX TS-program = xelatex

%\documentclass[11pt, a4paper]{article}
\documentclass[parskip, DIV=14]{scrartcl}
\usepackage{fontspec}
\usepackage{xltxtra} % a few fixes and extras
\usepackage{fancyhdr} % header

\usepackage{xcolor} % writing in color

\renewcommand{\baselinestretch}{1.2}% interligne + 20%

%\setmainfont[Ligatures=TeX]{Linux Libertine}
%\setmainfont[Ligatures=TeX]{Sanskrit 2003}
\setmainfont[Ligatures=TeX]{Chandas}

% footnote in end of document
\interfootnotelinepenalty=10000

\renewcommand{\thefootnote}{\fnsymbol{footnote}} %A sequence of nine symbols, try it and see!


% add a small title: name of the file + initials
\pagestyle{fancy}
\lhead{\color{lightgray} śrī gaṇapati upaniṣat}
\rhead{\color{lightgray} LSN}

\setkomafont{pagenumber}{\normalfont\bfseries}
\pagenumbering{roman}
\begin{document}

%\pagenumbering{gobble} % remove page numebr

\vspace{-1.5cm}

\begin{center}
% see command for footmark without numbering
\textbf{{\Huge॥\,~śrī gaṇapati upaniṣat\,~॥\LARGE\let\thefootnote\relax\footnote{\color{lightgray} Document written using \XeLaTeX{} and chandas font,  version 05/05/2016, LSN.}}}
\end{center}
\Large

\centering	
%\raggedright


॥\,~śrī ga॒ṇapatyatharvaśīrṣopaniṣat॒\,~॥\\
\vspace{0.2cm}
॥\,~śrī ma॒hā gaṇapati upaniṣat॒\,~॥ \\

\vspace{1cm}

॥\,~śānti pāṭhaḥ\,~॥ \\
\vspace{0.5cm}


oṁ bha॒draṁ karṇe̍bhiśśṛnu॒yāma̍ devāḥ~। \\
bha॒draṁ pa̍śyemā॒kṣabhi॒ryaja̍trāḥ~। sthi॒rairaṅgai̎stuṣṭu॒vāgṁsa̍sta॒nūbhi̍ḥ~। vyaśe̍ma de॒vahi̍ta॒ṁ yadāyu̍ḥ~।
sva॒sti na॒ indro̍ vṛ॒ddhaśra̍vāḥ~। sva॒sti na̍ḥ pū॒ṣā vi॒śvave̍dāḥ~। sva॒stina॒stārkṣyo॒ ari̍ṣṭanemiḥ~। sva॒sti no॒ bṛha॒spati̍rdadhātu\,~॥\\
 oṁ śānti॒śśānti॒śśānti̍ḥ\,~॥ 

$[$tanmāma̍vatu~। tadva॒ktāra̍mavatu\,~।\\
ava̍tu॒ mām~।
ava̍tu va॒ktāram̎\,~।\\
oṁ śānti॒śśānti॒śśānti̍ḥ~।\\

\vspace{0.8cm}

oṁ śrī gaṇeśāya॒ nama̍ḥ\,~॥$]$ \\

\vspace{1cm}

॥\,~gaṇapati artharva śīrṣa\,~॥ \\

\vspace{0.5cm}

oṁ nama̍ste ga॒ṇapa̍taye~। \\

tvame॒va pra॒tyakṣa॒ṁ tattva̍masi~।  \\
tvame॒va ke॒vala॒ṁ kartā̍’si~। \\
tvame॒va ke॒vala॒ṁ dhartā̍’si~। \\
tvame॒va ke॒vala॒gṁ॒ hartā̍’si~। \\
tvameva sarvaṁ khalvidaṁ̎ brahmā॒si~।  \\
tvagṁ sākṣādātmā̍’si ni॒tyam\,~।\\

\vspace{1cm}

ṛ̍taṁ va॒cmi~। sa̍tyaṁ va॒cmi\,~॥ \\

\vspace{0.5cm}

a॒va tva॒ṁ mām~। ava̍ va॒ktāram̎~। ava̍ śro॒tāram̎~। ava̍ dā॒tāram̎~। ava̍ dhā॒tāram̎~।
avānūcānama̍va śi॒ṣyam~। \\ ava̍ pa॒ścāttā̎t~।
ava̍ pu॒rastā̎t~। 
avotta॒rāttā̎t~।
ava̍ dakṣi॒ṇāttā̎t~।
ava̍ co॒rdhvāttā̎t~।
avādha॒rāttā̎t\,~।\\
sarvato māṁ pāhi pāhi̍ sama॒ntāt~॥\\

\vspace{1cm}
॥\,~svarūpa tattva\,~॥ \\
\vspace{0.5cm}

tvaṁ vāṅmaya̍stvaṁ cinma॒yaḥ~।
tvamānandamaya̍stvaṁ brahma॒mayaḥ~।
tvagṁ saccidānandā’dvi̍tīyo॒’si~।
tvaṁ pra॒tyakṣa॒ṁ brahmā̍si~।
tvaṁ jñānamayo vijñāna̍mayo॒’si\,~॥\\

\vspace{0.5cm}

sarvaṁ jagadidaṁ tva̍tto jā॒yate~। sarvaṁ jagadidaṁ tva̍ttasti॒ṣṭhati~। sarvaṁ jagadidaṁ tvayi laya̍meṣya॒ti~।
sarvaṁ jagadidaṁ tvayi̍ pratye॒ti~। tvaṁ bhūmirāpo’nalo’ni̍lo na॒bhaḥ~। tvaṁ catvāri vā̎kpadā॒ni\,~॥\\

\vspace{0.5cm}

tvaṁ gu॒ṇatra̍yātī॒taḥ\,~।
tvam avasthātra̍yātī॒taḥ\,~।\\
tvaṁ de॒hatra̍yātī॒taḥ\,~।
tvaṁ kā॒latra̍yātī॒taḥ\,~।\\
tvaṁ mūlādhārasthito̍’si ni॒tyam~। 
tvagṁ śaktitra̍yātma॒kaḥ\,~।\\
tvāṁ yogiṇo dhyāya̍nti ni॒tyam~।
tvaṁ brahmā tvaṁ viṣṇustvagṁ
rudrastvamindrastvamagnistvaṁ vāyustvagṁ sūryastvaṁ candramāstvaṁ brahma॒
bhūrbhuva॒ssvarom\,~॥\\

\vspace{1cm}
॥\,~nirguṇa upāsāna\,~॥ \\
\vspace{0.5cm}
॥\,~ganeśa vidyā\,~॥ \\
\vspace{0.5cm}

ga॒ṇādiṁ̎ pūrva̍muccā॒rya॒ va॒rṇādī̎gstada॒nantaram\,~।\\ 
anusvāraḥ pa̍rata॒raḥ~।
ardhe̎ndula॒sitam~।
tāre̍ṇa ṛ॒ddham~।
etattava manu̍svarū॒pam\,~।\\ 
gakāraḥ pū̎rvarū॒pam~।
 akāro madhya̍marū॒pam~। anusvāraścā̎ntyarū॒pam~। bindurutta̍rarū॒pam~। nāda̍ssandhā॒nam~।
  sagṁhi̍tā sa॒ndhiḥ~।
saiṣā gaṇe̍śavi॒dyā\,~॥\\ 

\vspace{1cm}
॥\,~a॒tha saṁkalpa॒ḥ\,~॥ \\
\vspace{0.5cm}

gaṇa̍ka ṛ॒ṣiḥ\,~।\\ nicṛdgāya̍trīccha॒ndaḥ\,~।\\ 
$[$śrī mahā$]$ gaṇapati̍rdeva॒tā\,~।\\
oṁ gaṁ ga॒ṇapa̍taye namaḥ\,~॥\\

\vspace{1cm}
॥\,~gaṇapati gāyatrī\,~॥ \\

\vspace{0.4cm}

om ekada॒ntāya̍ vi॒dmahe̍ vakratu॒ṇḍāya̍ dhīmahi\,~।\\ tanno̍ dantiḥ praco॒dayā̎t\,~॥\\

\vspace{1cm}
॥\,~saguṇa upāsāna\,~॥ \\
\vspace{0.5cm}
॥\,~ganeśa rūpa\,~॥ \\
\vspace{0.5cm}

$[$om$]$ ekada॒ntaṁ ca̍turha॒sta॒ṁ pā॒śama̍ṅkuśa॒dhāri̍ṇam\,~।\\ rada̍ṁ ca॒ vara̍dagṁ ha॒stai॒rbibhrāṇa̍ṁ mūṣa॒kadhva̍jam\,~।\\ rakta̍ṁ la॒mboda̍ragṁ śū॒rpa॒ka॒rṇakagṁ̍ rakta॒vāsa̍sam\,~।\\ rakta̍ga॒ndhānu̍liptā॒ṅga॒gṁ॒ ra॒ktapu̍ṣpaissu॒pūji̍tam\,~।\\ bhaktā̍nu॒kampi̍naṁ de॒va॒ṁ ja॒gatkā̍raṇa॒macyu̍tam\,~।\\ āvi̍rbhū॒taṁ ca̍ sṛ॒ṣṭyā॒dau॒ pra॒kṛte̎ḥ puru॒ṣātpa̍ram\,~।\\
eva̍ṁ dhyā॒yati̍ yo ni॒tya॒gṁ॒ sa॒ yogī̍ yogi॒nāṁ va̍raḥ\,~॥\\

\vspace{1cm}
॥\,~aṣṭa nāma gaṇapati\,~॥ \\
\vspace{0.4cm}

$[$oṁ$]$ namo vrātapataye\,~।\\
$[$oṁ$]$ namo gaṇapataye\,~।\\
$[$oṁ$]$ namaḥ pramathapataye\,~।\\
$[$oṁ$]$ namaste’stu lambodarāyaikadantāya vighnanāśine śivasutāya varadamūrtaye॒ namaḥ\,~॥\\

\vspace{1cm}
॥\,~phala śruti\,~॥ \\
\vspace{0.5cm}

etadatharvaśīrṣa̍ṁ yo’dhī॒te sa brahmabhūyā̍ya ka॒lpate\,~।\\ sa sarvavighnai̎rna bā॒dhyate\,~।\\
sa sarvatra sukha̍medha॒te\,~।\\
sa pañcamahāpāpā̎t pramu॒cyate\,~।\\
sā॒yama̍dhīyā॒no॒ divasakṛtaṁ pāpa̍ṁ nāśa॒yati\,~।\\ prā॒tara̍dhīyā॒no॒ rātrikṛtaṁ pāpa̍ṁ nāśa॒yati\,~।\\ sāyaṁ prātaḥ pra̍yuñjā॒no॒ pāpo’pā̍po bha॒vati\,~।\\ sarvatrādhīyāno’pavi̍ghno bha॒vati\,~।\\ 
dharmārthakāmamokṣa̍ṁ ca vi॒ndati\,~।\\ idamatharvaśīrṣamaśiṣyāya̍ na de॒yam\,~।\\
yo yadi mo̍hāddā॒syati sa pāpī̍yān bha॒vati\,~।\\ 
sahasrāvartanādyaṁ yaṁ kāma̍madhī॒te taṁ tamane̍na
sā॒dhayet\,~॥\\

\vspace{1cm}

anena gaṇapatima̍bhiṣi॒ñcati sa vā̎gmī bha॒vati\,~।\\
caturthyāmana̍śnan ja॒pati sa vidyā̍vān bha॒vati\,~।\\
ityatharva̍ṇavā॒kyam\,~।\\
brahmādyā॒vara̍ṇaṁ vi॒dyānna bibheti kadā̍cane॒ti\,~॥\\

\vspace{1cm}

yo dūrvāṅku̍rairya॒jati sa vaiśravaṇopa̍mo bha॒vati\,~।\\ yo lā̍jairya॒jati sa yaśo̍vān bha॒vati\,~।\\
sa medhā̍vān bha॒vati\,~।\\
yo modakasahasre̍ṇa ya॒jati sa vāñcchitaphalama̍vāpno॒ti\,~।\\ 
yassājyasami̍dbhirya॒jati sa sarvaṁ labhate sa sa̍rvaṁ
la॒bhate\,~॥\\

\vspace{0.5cm}

aṣṭau brāhmaṇān samyag grā̍hayi॒tvā sūryavarca̍svī bha॒vati\,~।\\
sūryagrahe ma̍hāna॒dyāṁ pratimāsannidhau vā ja॒ptvā siddhama̍ntro bha॒vati\,~।
mahāvighnā̎t pramu॒cyate\,~।\\ mahādoṣā̎t pramu॒cyate\,~।
 mahāprātyavāyā̎t pramu॒cyate\,~।\\
sa sarvavidbhavati sa sarva̍vid bha॒vati\,~।\\ 
ya e̍vaṁ ve॒da\,~।
ityu̍pa॒niṣa̍t\,~॥\\

\vspace{1cm}

॥\,~śānti pāṭhaḥ\,~॥ \\

\vspace{0.5cm}


oṁ bha॒draṁ karṇe̍bhiśśṛnu॒yāma̍ devāḥ~।\\
 bha॒draṁ pa̍śyemā॒kṣabhi॒ryaja̍trāḥ~। sthi॒rairaṅgai̎stuṣṭu॒vāgṁsa̍sta॒nūbhi̍ḥ~। vyaśe̍ma de॒vahi̍ta॒ṁ yadāyu̍ḥ~।
sva॒sti na॒ indro̍ vṛ॒ddhaśra̍vāḥ~। sva॒sti na̍ḥ pū॒ṣā vi॒śvave̍dāḥ~। sva॒stina॒stārkṣyo॒ ari̍ṣṭanemiḥ~। sva॒sti no॒ bṛha॒spati̍rdadhātu\,~।\\
 oṁ śānti॒śśānti॒śśānti̍ḥ\,~॥ 

\vspace{0.6cm}

oṁ sa॒ha nā̍vavatu\,~।
sa॒ha nau̍ bhunaktu\,~।\\
sa॒ha vī॒rya̍ṁ karavāvahai\,~।
te॒ja॒svinā॒vadhī̍tamastu॒\,~।\\
mā vi̍dviṣā॒vahai̎\,~।\\
oṁ śānti॒śśānti॒śśānti̍ḥ\,~॥

\vspace{1cm}
॥\,~iti ga॒ṇapatyarthaśīrṣopaniṣat samā॒ptā\,~॥ \\

\end{document}