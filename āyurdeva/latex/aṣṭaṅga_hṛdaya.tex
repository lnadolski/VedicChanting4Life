%!TEX encoding = UTF-8 Unicode
%!TEX TS-program = xelatex

%\documentclass[11pt, a4paper]{article}
\documentclass[parskip, DIV=14]{scrartcl}
\usepackage{fontspec}
\usepackage{xltxtra} % a few fixes and extras
\usepackage{fancyhdr} % header

\usepackage{xcolor} % writing in color

\renewcommand{\baselinestretch}{1.2}% interligne + 20%

%\setmainfont[Ligatures=TeX]{Linux Libertine}
%\setmainfont[Ligatures=TeX]{Sanskrit 2003}
%\setmainfont[Ligatures=TeX]{uttara}
\setmainfont[Ligatures=TeX]{Chandas}

% footnote in end of document
\interfootnotelinepenalty=10000

\renewcommand{\thefootnote}{\fnsymbol{footnote}} %A sequence of nine symbols, try it and see!


% add a small titleḥ name of the file + initials
\pagestyle{fancy}
\lhead{\color{lightgray} aṣṭaṅga hṛdaya}
%\lhead{\color{lightgray} īśopaniṣat}
\rhead{\color{lightgray} LSN}

\setkomafont{pagenumber}{\normalfont\bfseries}
\pagenumbering{roman}
\begin{document}

%\pagenumbering{gobble} % remove page numebr

\vspace{-1.5cm}

\begin{center}
% see command for footmark without numbering
\textbf{{\Huge॥\,~aṣṭaṅga hṛdaya~॥  \LARGE\let\thefootnote\relax\footnote{\color{lightgray} Document written using \XeLaTeX{} and chandas font,  version 04/05/2019, LSN.}}}
\end{center}
\Large

% only  Upanisad that forms an integral part of a vedic Samhita
%Notation of the śukla yajurveda
% cf śrī sūktam

% two recensionsḥ  the Madhyandina and the Kanva

% īśaḥ the Creator, the Supreme Intelligence

% mantropaniṣad, since it come from the end of the saṁhita of the śukla yajurveda
% Most upaniṣad comes from the brahmaṇa part of the veda and are called brahmopaniṣad

{\centering	
%\raggedright

%%%%%%%%%%%%%%%%%% Gḥāṇā %%%%%%%%%%%%%%%%%%%%%

\large

\Large
॥\,~śānti pāṭhaḥ\,~॥\\
\vspace{0.5cm}

rāgādi-rogān satatānuṣaktān a-śeṣa-kāya-prasṛtān a-śeṣān | 
autsukya-mohā-rati-dāñ jaghāna yo '-pūrva-vaidyāya namo 'stu tasmai || 1 || 

āyuḥ-kāmayamānena dharmārtha-sukha-sādhanam | 
āyur-vedopadeśeṣu vidheyaḥ param ādaraḥ || 2 || 

brahmā smṛtvāyuṣo vedaṃ prajāpatim ajigrahat | 
so 'śvinau tau sahasrākṣaṃ so 'tri-putrādikān munīn || 3 || 

te 'gniveśādikāṃs te tu pṛthak tantrāṇi tenire | 
tebhyo 'ti-viprakīrṇebhyaḥ prāyaḥ sāra-taroccayaḥ || 4 || 

kriyate 'ṣṭāṅga-hṛdayaṃ nāti-saṃkṣepa-vistaram | 
kāya-bāla-grahordhvāṅga-śalya-daṃṣṭrā-jarā-vṛṣān || 5 ||

1.5bv nāti-saṃkṣipta-vistṛtam aṣṭāv aṅgāni tasyāhuś cikitsā yeṣu saṃśritā | vāyuḥ pittaṃ kaphaś ceti trayo doṣāḥ samāsataḥ || 6 || vikṛtā-vikṛtā dehaṃ ghnanti te vartayanti ca | te vyāpino 'pi hṛn-nābhyor adho-madhyordhva-saṃśrayāḥ || 7 ||

1.7bv ghnanti te vardhayanti ca vayo-'ho-rātri-bhuktānāṃ te 'nta-madhyādi-gāḥ kramāt |
āyuḥ kāmāyamānena dharmārtha sukhasādhanam | 
āyurvedopadeśeṣu vidheyaḥ paramādaraḥ

kāyabālagrahordhvāṅga śalyadaṃṣṭrā jarāvṛṣān || aṣṭāvaṅgāni tasyāhuḥ cikitsā yeṣu saṃśritā

vāyuḥ pittaṃ kaphaśceti trayo doṣāḥ samāsataḥ || vikṛtā’vikṛtā dehaṃ ghnanti te varttayanti ca 

\Large
%%%%%%%%%%%%%%%%%%%%%%%%%%%%%%%%%%%%%%%%%%
% visible / unvisible 

\end{document}