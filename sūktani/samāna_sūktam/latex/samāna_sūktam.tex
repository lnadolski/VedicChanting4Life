%!TEX encoding = UTF-8 Unicode
%!TEX TS-program = xelatex

%\documentclass[11pt, a4paper]{article}
\documentclass[parskip, DIV=18]{scrartcl}
\usepackage{fontspec}
\usepackage{xltxtra} % a few fixes and extras

\usepackage{xcolor} % writing in color


%\setmainfont[Ligatures=TeX]{Linux Libertine}
%\setmainfont[Ligatures=TeX]{Sanskrit 2003}
\setmainfont[Ligatures=TeX]{Chandas}

% Requires font Nakula by John Smith,
% see http://bombay.indology.info/software/fonts/devanagari/indexḥtml
\newcommand\skt{\catcode`\~=12
           \fontspec[Script=Devanagari,Mapping=velthuis-sanskrit]{Chandas}}
\newfontfamily
  \sanskritfont [Script=Devanagari,Mapping=velthuis-sanskrit]{Sanskrit 2003}
%  \sanskritfont [Script=Devanagari,Mapping=velthuis-sanskrit]{Sahadeva}
%\sanskritfont [Script=Devanagari,Mapping=velthuis-sanskrit]{Devanagari Sangam ṁN}

\renewcommand{\thefootnote}{\fnsymbol{footnote}} %A sequence of nine symbols, try it and see!

\begin{document}


\pagenumbering{gobble} % remove page numebr

\vspace{-1.5cm}

\begin{center}
\textbf{{\Huge ॥\,samāna sūktam\,॥} \LARGE\let\thefootnote\relax\footnote{ \color{lightgray} Document written using \XeLaTeX{} and chandas font, LSN, version 17/01/2016.}}
\end{center}
\Large

\centering
%\raggedright

\vspace{0.5cm}


॥\, HYMN 191 (RV Book 10) \,॥


oṁ saṁsa॒midyu̍vase vṛṣa॒nnagne॒ viśvā̎nya॒rya ā\,। \\
i॒ḷaspa॒desami̍dhyase॒ sa no॒ vasū॒nyā bha̍ra  ॥\,1\,॥ \par

saṁ ga̍tchadhva॒ṁ saṁ va̍dadhva॒ṁ saṁ vo॒ manā̎ṁsi jānatām\,। \\
de॒vā bhā॒gaṁ yathā॒ pūrve̎ saṁjānā॒nā u॒pāsa̍te  ॥\,2\,॥ \par

sa॒mā॒no mantra॒ḥ sami̍tiḥ samā॒nī sa̍mā॒naṁ mana̍ḥ sa॒ha ci॒ttame̎ṣām\,। \\
sa॒mā॒naṁ mantra̍ma॒bhi man̎traye vaḥ samā॒nena̍ voha॒viṣā̎ juhomi  ॥\,3\,॥ \par

sa॒mā॒nī va॒ ākū̎tiḥ samā॒nā hṛda̍yāni vaḥ\,। \\
sa॒mā॒nama̍stu vo॒mano॒ yathā̎ va॒ḥ susa॒hāsa̍ti  ॥\,4\,॥ \par

%%\vspace{0.5cm}

 
\vspace{0.5cm}
 
\begin{center}
 ॥\,oṁ śānti॒ḥ śānti॒ḥ śānti̍ḥ\,॥
\end{center}

\vspace{1.5cm}

\end{document}