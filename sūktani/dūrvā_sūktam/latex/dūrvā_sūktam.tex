%!TEX encoding = UTF-8 Unicode
%!TEX TS-program = xelatex

%\documentclass[11pt, a4paper]{article}
\documentclass[parskip, DIV=18]{scrartcl}
\usepackage{fontspec}
\usepackage{xltxtra} % a few fixes and extras

\usepackage{xcolor} % writing in color


%\setmainfont[Ligatures=TeX]{Linux Libertine}
%\setmainfont[Ligatures=TeX]{Sanskrit 2003}
\setmainfont[Ligatures=TeX]{Chandas}

% Requires font Nakula by John Smith,
% see http://bombay.indology.info/software/fonts/devanagari/indexḥtml
\newcommand\skt{\catcode`\~=12
           \fontspec[Script=Devanagari,Mapping=velthuis-sanskrit]{Chandas}}
\newfontfamily
  \sanskritfont [Script=Devanagari,Mapping=velthuis-sanskrit]{Sanskrit 2003}
%  \sanskritfont [Script=Devanagari,Mapping=velthuis-sanskrit]{Sahadeva}
%\sanskritfont [Script=Devanagari,Mapping=velthuis-sanskrit]{Devanagari Sangam MN}

\renewcommand{\thefootnote}{\fnsymbol{footnote}} %A sequence of nine symbols, try it and see!

\begin{document}


\pagenumbering{gobble} % remove page numebr

\vspace{-1.5cm}

\begin{center}
\textbf{{\Huge ॥\,dūrvā sūktam\,॥} \LARGE\let\thefootnote\relax\footnote{ \color{lightgray} Document written using \XeLaTeX{} and chandas font, LSN, version 17/01/2016.}}
\end{center}
\Large

\centering
%\raggedright

\vspace{0.5cm}

sa॒ha॒sra॒para̍mā de॒vī॒ śa॒tamū̍lā śa॒tāṅku̍rā | 
sarvagṁ̍ haratu̍ me
pā॒pa॒ṁ dū॒rvā du̍svapna॒nāśa̍nī ॥\,1\,॥ \par

kāṇḍā̎t kāṇḍāt pra॒roha̍ntī॒ paru̍ṣaḥ paruṣa॒ḥ pari̍ | e॒vā no̍
dūrve॒ prata̍nu sa॒hasre̍ṇa śa॒tena̍ ca ॥\,2\,॥ \par

yā śa॒tena̍ prata॒noṣi̍ sa॒hasre̍ṇa vi॒roha̍si | tasyā̎ste devīṣṭake
vi॒dhema̍ ha॒viṣā̍ va॒yam ॥\,3\,॥ \par

aśva̍krā॒nte ra̍thakrā॒nte॒ vi॒ṣṇukrā̎nte va॒sundha̍rā | śirasā̍
dhāra̍yiṣyā॒mi॒ ra॒kṣa॒sva mā̎ṁ pade॒pade  ॥\,4\,॥ \par
%%\vspace{0.5cm}

 
\vspace{0.5cm}
 
\begin{center}
 ॥\,oṁ śānti॒ḥ śānti॒ḥ śānti̍ḥ\,॥
\end{center}

\vspace{1.5cm}

\end{document}