%!TEX encoding = UTF-8 Unicode
%!TEX TS-program = xelatex

%\documentclass[11pt, a4paper]{article}
\documentclass[parskip, DIV=14]{scrartcl}
\usepackage{fontspec}
\usepackage{xltxtra} % a few fixes and extras
\usepackage{fancyhdr} % header

\usepackage{xcolor} % writing in color

\renewcommand{\baselinestretch}{1.2}% interligne + 20%

%\setmainfont[Ligatures=TeX]{Linux Libertine}
%\setmainfont[Ligatures=TeX]{Sanskrit 2003}
%\setmainfont[Ligatures=TeX]{uttara}
\setmainfont[Ligatures=TeX]{Chandas}

% footnote in end of document
\interfootnotelinepenalty=10000

\renewcommand{\thefootnote}{\fnsymbol{footnote}} %A sequence of nine symbols, try it and see!


% add a small titleḥ name of the file + initials
\pagestyle{fancy}
\lhead{\color{lightgray} navagraha sūktam}
%\lhead{\color{lightgray} īśopaniṣat}
\rhead{\color{lightgray} LSN}

\setkomafont{pagenumber}{\normalfont\bfseries}
\pagenumbering{roman}
\begin{document}

%\pagenumbering{gobble} % remove page numebr

\vspace{-1.5cm}

\begin{center}
% see command for footmark without numbering
\textbf{{\Huge॥\,~navagraha sūktam~॥  \LARGE\let\thefootnote\relax\footnote{\color{lightgray} Document written using \XeLaTeX{} and chandas font,  version 14/07/2019, LSN.}}}
\end{center}
\Large

{%\centering	
%\raggedright

%ṛgvedaḥ: mam.-10, sū.-163, ṛṣiḥ vivṛhā kāśyapaḥ yakṣmanāśanam anuṣṭup

% smṛti: notation from K. tradition
śu̍klā̍ṁba̍radha̍raṁ vi̱ṣṇuṁ śa̍śi̍va̍rṇaṁ ca̍turbhu̱ja̱m~|
pra̍sa̍nna̍vada̍naṁ dhyā̱yet sa̍rva̍ vi̍ghnopa̍śānta̱ye॒~||

oṁ bhūḥ~| oṁ bhuva̍ḥ~| ogṁ॒ suva̍ḥ~| oṁ maha̍ḥ~| oṁ jana̍ḥ~| oṁ tapa̍ḥ~| ogṁ̍ sa॒tyam~|

oṁ tatsa̍vi॒turvare̎ṇya॒ṁ bhargo̍ de॒vasya̍ dhīmahi~|
dhiyo॒ yo na̍ḥ praco॒dayā̎t~| omāpo॒ jyotī॒raso॒'mṛta॒ṁ brahma॒ bhūrbhuva॒ssuva॒rom~||

\vspace{0.5cm}

%atha saṅkapaḥ~||

mamopāttasamastaduritakṣa̍yadvā॒rā śrīpara̍meśvaraprītyartham ādi̍tyādinavagrahadevatā-prasā̍dasiddhyartham ādi̍tyādi navagrahamaskārā̍n kari॒ṣye~||

%Other translation. See Veda mantraās and sūktaās by R. L. Kashyapa

%les 9 influences astrales, comprenant le Soleil [Sūrya], la Lune [Candra], les 5 planètes Mars [Aṅgāraka], Mercure [Budha], Jupiter [Bṛhaspati], Vénus [Śukra] et Saturne [Śani], et les deux démons de l'éclipse Rāhu et de la comète Ketu; on les représente souvent sculptés sur le linteau du sanctuaire d'un temple.

% Nine plate of the subtil world (sūkṣma jagat) / planet of the mental world / force to invoke and develop inside ourselves
% See Aurobindu insights on the planet

\vspace{0.5cm}

%%SUN/SUNDAY āditya

om āsa̱tyena̱ raja̍sā̱ varta̍māno nive̱śaya̍nna̱mṛta̱ṁ martya̍ñca~| 
hi̱ra̱ṇyaye̍na savi̱tā rathe̱nā''de̱vo yā̍ti̱ bhuva̍nā vi̱paśyan̍~||
 
%With the Light of Truth in space advancing, determining life and death, borne in his golden chariot he comes, Savitar, God who gazes upon the worlds. (Rig Veda 1.35.2 ; Taitt. Sam.3.4.11.2a)

a̱gniṁ dū̱taṁ vṛ̍ṇīmahe̱ hotā̍raṁ vi̱śvave̍dasam~| 
a̱sya ya̱jñasya̍ su̱kratum̎~||

%We choose Agni as our messenger, the herald, master of all wealth. Well skilled in this our sacrifice. (Rig Veda 1.12.1; Taitt. Sam. 2.5.8.5)

yeṣā̱mīśe̍ paśu̱pati̍ḥ paśū̱nāṁ catu̍ṣpadāmu̱ta ca̍ dvi̱padā̎m~| 
niṣkrī̍to̱’yaṁ ya̱jñiya̍ṁ bhā̱game̍tu rā̱yaspoṣā̱ yaja̍mānasya santu~||\\

%Which creatures does the Lord of creatures rule:— both the four footed and birds. May He, being propitiated, accept His sacrificial share, may abundance of wealth come to the sacrificer. (T.S. 3;1;4d)

om adhidevatā pratyadhidevatā sahitāya ādi̍tyāya॒ nama̍ḥ~||~\,1\,~||
\vspace{0.5cm}
\newpage
%% MOON/ Monday/soma
om āpyā̍yasva̱ same̍tu te vi̱śvata̍ssoma̱ vṛṣṇi̍yam~| bhavā̱ vāja̍sya saṅga̱the~||

%Swell up (gonfle/grossi/augmente), O Soma! Let your strength be gathered from all sides. Be strong in the gathering of might. (Rig Veda 1;91;16 & T.S. 3;2;5K)

a̱psu me̱ somo̍ abravīda̱ntar-viśvā̍ni bheṣa̱jā~| 
a̱gniñca̍ vi̱śva śa̍ṁbhuva̱māpa̍śca vi̱śva bhe̍ṣajīḥ~||

%A skilled physician tells me, that in the waters of life lies the capacity to heal all ailments. In the fire of wisdom the welfare of the world and in the waters of life a panacea. (Atharva Veda 1.6.2.)

gau̱rī mi̍māya sali̱lāni̱ takṣa̱t-yeka̍padī dvi̱padī̱ sā catu̍ṣpadī~| 
a̱ṣṭā-pa̍dī̱ nava̍-padī babhū̱vuṣī̍ sa̱hasrā̎kṣarā para̱me vyo̍man~||

%The Vedas have spoken of various forms of knowledge and preached multifarious duties. It deals with one Supreme Godhead, it gives knowledge of the past and the future, It teaches of religion, prosperity, fulfillment of desires and salvation. It grants the eight siddhis, obtainable through the nine organs, through its thousands of words it leads to the highest Abode. (Rig Veda 1.164.41 & Atharva Veda 9.10.21)

om adhidevatā pratyadhidevatā sahitāya somā̍ya॒ nama̍ḥ~||~\,2\,~||
\vspace{0.5cm}

%% MARS/Tuesday aṅgāraka (mangal)
om a̱gnirmū̱rddhā di̱vaḥ ka̱kut-pati̍ḥ pṛthi̱vyā a̱yam~| a̱pāgṁ retāg̍ṁ si jinvati~||

%Agni manifests in three forms; as the Sun as lightening and as earthly fire. He activates the seed of life. (Rig Veda 8;54;16 & T.S. 1;5;5c)

syo̱nā pṛ̍thivi̱ bhavā̍'nṛkṣa̱rā ni̱veśa̍nī~| yacchā̍na̱ś-śarma̍ sa̱prathā̎ḥ~||

%May you be thornless O Earth, spread wide before us for a dwelling place. Grant us shelter broad and secure. (Rig Veda 1.22.15.)

kṣetra̍sya̱ pati̍nā va̱yagṁ hi̱tene̍va jayāmasi~| gāmaśva̍ṁ poṣayi̱ntvā sa no̍ mṛḍātī̱dṛśe̎~||

%Through the Lord of the Field, as from a friend, we obtain what nourishes our cattle & horses, in such may He be good to us. (Rig Veda 4.57.1.)

om adhidevatā pratyadhidevatā sahitāya aṅgā̍rakāya॒ nama̍ḥ~||~\,3\,~||
\vspace{0.5cm}

%JULY 27

%% WEDNESDAY Mercury

om udbu̍dhyasvāgne̱ prati̍jā gṛhyenamiṣṭā pū̱rte sagṁsṛ̍jethāma̱yañca̍~| 
puna̍ḥ kṛ̱ṇvaggstvā̍ pi̱tara̱ṁ yuvā̍nam-a̱nvātāgṁ̍ sī̱ttvayi̱ tantu̍m-e̱tam~||

%Awaken O Agni! O Light of wisdom! and keep us vigilant in the practice of works done for our own merit and works done for the welfare of all beings, may we remain together, making the Pitris young with life's renewal, the thread is being maintained through you. (Vajasaneyi Samhita. 15:55.)

i̱daṁ viṣṇu̱r-vica̍krame tre̱dhā nida̍dhe pa̱dam~| samū̍ḍham-asya pāgṁ su̱re~||

%Through all this world strode Vishnu; thrice His foot he planted, and the whole universe was gathered in His footstep's dust. (Rig Veda 1:22:17)

viṣṇo̍r-a̱rāṭa̍masi̱ viṣṇo̎ḥ pṛ̱ṣṭhama̍si̱ viṣṇo̱ḥ śñaptre̎stho̱ viṣṇo̱s-syūr-a̍si̱ viṣṇo̎rdhru̱vam-a̍si vaiṣṇa̱vam-a̍si̱ viṣṇa̍ve tvā~||

%You are the forehead of Vishnu; you are the back of Vishnu; you two are the corners of Vishnu's mouth. You are the thread of Vishnu; you are the fixed point of Vishnu;. you belong to Vishnu; to Vishnu you are offered. (Taittiriya Samhita 1:2:13)

om adhidevatā pratyadhidevatā sahitāya budhā̍ya॒ nama̍ḥ~||~\,4\,~||
\vspace{0.5cm}

%% Thursday/bṛhaspati Jupiter/Jeudi bṛha "the vast"
oṁ bṛha̍spate̱ ati̱yada̱ryo arhā̎ddyu̱mad-vi̱bhāti̱ kratu̍ma̱jjane̍ṣu~| yaddī̱daya̱cchava̍sarta prajāta̱ tad-a̱smāsu̱ dravi̍ṇan dhehi ci̱tram~||

%O Brhaspati, who are born of holy order; that Divine Wisdom shall overcome the enemies of the mind, that wisdom shall shine glorious, with insight among men. That wisdom shall be resplendent in glory. (Taittiriya Samhita 1;8;22 g)

indra̍ marutva i̱ha pā̍hi̱ soma̱ṁ yathā̍ śāryā̱te api̍bas-su̱tasya̍~| tava̱ praṇī̍tī̱ tava̍ śūra̱-śarma̱n-nāvi̍vā santi ka̱vaya̍s-suya̱jñāḥ~||

%O Indra surrounded by the Maruts drink here the Soma! As you did drink the juice beside the Saryata. Under your guidance, in your keeping, O Hero! the singers serve, skilled in fair sacrifice. (Vajasaneyi Samhita. 7:35.)

brahma̍ jajñā̱naṁ pra̍tha̱maṁ pu̱rastā̱d-visī̍ma̱tas-su̱ruco̍ ve̱na ā̍vaḥ~| subu̱dhniyā̍ upa̱mā a̍sya vi̱ṣṭhās-sa̱taśca̱ yoni̱m-asa̍taśca̱ viva̍ḥ~||

%In the beginning, first was the Veda generated, the delight of existence overcame the gods from on high revealing the most profound and simple revelations — the source of the existent and the non-­‐existent. (Vajasaneyi Samhita 13:3)

om adhidevatā pratyadhidevatā sahitāya bṛha॒spa̍te॒ nama̍ḥ~||~\,5\,~||
\vspace{0.5cm}

%% Friday/sukra Venus

oṁ prava̍ś-śu̱krāya̍ bhā̱nave̍ bharadhvam~| ha̱vyaṁ ma̱tiṁ cā̱gnaye̱ supū̍tam~| yo daivyā̍ni̱ mānu̍ṣā ja̱nūgṁṣi̍~| a॒ntar-viśvā̍ni vi॒dma nā॒ jigā̍ti~||

%Bring forth your offerings to his refulgent splendour; your hymn as purest offering to Agni the mystic fire of wisdom who goes as messenger conveying all songs of men to the gods in heaven. (Rig Veda 7.4.1.)

i̱ndrā̱ṇīm-ā̱su nāri̍ṣu su̱patnī̍m-a̱ham-a̍śravam~| na hya̍syā apa̱rañca̱na ja̱rasā̱ mara̍te̱ pati̍ḥ~||

%So have I heard Indrani called the most fortunate from amongst women. For never shall her consort die in future time, through old age. (Rig Veda 10.86.11. & T.S.1.7.13.1.)

indra̍ṁ vo vi̱śvata̱spari̱ havā̍mahe॒ jane̎bhyaḥ~| a̱smāka̍m-astu keva̍laḥ~||

%O Indra ruler of the universe we invoke you from amongst others. Favour us alone. (T.S;1;6;12. Rig Veda 1;7;10)

om adhidevatā pratyadhidevatā sahitāya sukrā̍ya॒ nama̍ḥ~||~\,6\,~||
\vspace{0.5cm}

%% Saturday /Saturn  śani / śanaiścara "which move slowly"
oṁ śanno̍ de̱vīra̱bhiṣṭa̍ya॒ āpo̍ bhavantu pī̱taye̎~| śaṁyor-a̱bhisra̍vantu naḥ~||

%May the seven cosmic Principles be propitious for us; divine forces for our aid & bliss. Let them flow for us, for health and strength. (Rig Veda 10.9.4. & A.tharva Veda 1.6.1.)

prajā̍pate̱ na tvade̱tānya̱nyo viśvā̍ jā̱tāni̱ pari̱tā ba̍bhūva~| yat kā̍māste juhu̱mastanno̍ astu va̱yaggsyā̍ma̱ pata̍yo rayī̱ṇām~||

%O Lord of Beings, you alone can comprehend all these created forms, and none beside you. Grant us our heart's desire when we invoke you, may we become lords of valuable possessions. (Vajasaneyi Samhita 10;20)

i̱maṁ ya̍ma prasta̱ramā hi sīdāṅgi̍robhiḥ pi̱tṛbhi̍ḥ saṁvidā̱naḥ~| ātvā̱ mantrā̎ḥ kaviśa̱stā va̍hantve̱nā rā̍jan ha̱viṣā̍ mādayasva~||

%O Yama! Come and be seated in this place, in company with the manes. Let the hymns recited by the sages convey you O King, be gladdened by this oblation. (Rig Veda 10.14.4.)

om adhidevatā pratyadhidevatā sahitāya śanaiśca̍rāya॒ nama̍ḥ~||~\,7\,~||
\vspace{0.5cm}

%% rāhu
oṁ kayā̍ naści̱tra ābhu̍vadū̱tī sa̱dāvṛ̍dha̱s-sakhā̎~| kayā̱ śaci̍ṣṭhayā vṛ̱tā~||

%What sustenance will he bring to us, wonderful ever prospering friend? With what most mighty company. (S.Y.V. 27:39)

ā'yaṅ-gauḥ pṛśni̍rakramī̱dasa̍nan-mā̱tara̱ṁ puna̍ḥ~| pi̱tara̍ñca pra̱yant-suva̍ḥ~||

%The Godhead has appeared as this variegated universe along with Mother Nature. Advancing towards the Highest heaven. (Rig Veda X :189:1)

yatte̍ de̱vī niṛṛ̍tir-āba̱bandha̱ dāma̍ grī̱vāsva̍vica̱rtyam~| i̱daṁ te̱ tad-viṣyā̱myāyu̍ṣo̱ na madhyā̱dathā̍ jī̱vaḥ pi̱tuma̍ddhi̱ pramu̍ktaḥ~||

%O man that noose of suffering that is fastened around your neck, hard to loosen, I release, so that you may attain long life and prosperity and enjoyment. (Taittiriya Samhita 4.2.5.2.)

om adhidevatā pratyadhidevatā sahitāya rāha̍ve॒ nama̍ḥ~||~\,8\,~||
\vspace{0.5cm}

%% ketu

oṁ ke̱tuṁkṛ̱vanna̍ ke̱tave̱ peśo̍ maryā ape̱śase̎~| samu̱ṣadbhi̍r-ajāyathāḥ~||

%Making a banner for that which has none, Form for the formless, O you men, you were born with the dawn. (Taittiriya Samhita 7;4;20h)

bra̱hmā de̱vānā̎ṁ pada̱vīḥ ka̍vī̱nāmṛṣi̱r-viprā̍ṇāṁ mahi̱ṣo mṛ̱gāṇā̎m~| śye̱no gṛdhrā̍ṇā̱gṁ॒ svadhi̍ti̱r vanā̍nā̱gṁ॒ soma̍ḥ pa̱vitra̱m-atye̍ti̱ rebhan̍~||

%Brahma of the gods, leader of poets, Sage of seers, bull of wild beasts. Eagle of vultures, axe of the forests, Soma goes over the seive singing. (Taittiriya Samhita 3;4;11d)

saci̍tra ci̱traṁ citayan̎ tama̱sme citra̍ kṣatra ci̱trata̍maṁ vayo̱dhām~| ca̱ndraṁ ra̱yiṁ pu̍ru̱vīraṁ̎ bṛ̱hanta̱ṁ candra̍ ca̱ndrābhi̍rgṛṇa̱te yu̍vasya~||

om adhidevatā pratyadhidevatā sahitāya ketu̍bhyo॒ nama̍ḥ~||~\,9\,~||
\vspace{0.5cm}

%Wondrous! Of wondrous power! I give to the singer wealth wondrous, outstanding, most wonderful, life-­giving. Bright wealth, O Refulgent Divine Wisdom, vast, with many aspects, give understanding to your devotee. (Rig Veda 6.6.7.)
\vspace{1.0cm}

\begin{center}
 ॥\,oṁ śānti॒ḥ śānti॒ḥ śānti̍ḥ\,॥
 \end{center}

\Large
%%%%%%%%%%%%%%%%%%%%%%%%%%%%%%%%%%%%%%%%%%
% visible / unvisible 

\end{document}