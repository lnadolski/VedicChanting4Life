%!TEX encoding = UTF-8 Unicode
%!TEX TS-program = xelatex

%\documentclass[11pt, a4paper]{article}
\documentclass[parskip, 12pt, DIV=16, pagenumber=head,top, enableddeprecatedfontcommands]{scrartcl}
%\documentclass[parskip, DIV=16, pagenumber=head,top, enableddeprecatedfontcommands]{scrartcl}
\usepackage{fancyhdr} % header
\usepackage{fontspec}
\usepackage{xltxtra} % a few fixes and extras
%{enableddeprecatedfontcommands}

\usepackage{xcolor} % writing in color

\renewcommand{\baselinestretch}{1.2}% interligne + 20%

\setmainfont[Ligatures=TeX]{Sanskrit 2003}
%\setmainfont[Ligatures=TeX]{Chandas}

% footnote in end of document
\interfootnotelinepenalty=10000

\renewcommand{\thefootnote}{\fnsymbol{footnote}} %A sequence of nine symbols, try it and see!

% add a small title: name of the file + initials
\pagestyle{fancy}
\lhead{\color{lightgray} dṛgdṛśyaviveka}
\rhead{\color{lightgray} LSN}

\setkomafont{pagenumber}{\normalfont\bfseries}
\pagenumbering{roman}

\begin{document}

%\pagenumbering{gobble} % remove page numebr

\vspace{-1.5cm}

\begin{center}
% see command forfootmark without numbering
\textbf{{\Huge॥\,~śrī dṛ॒gdṛśyaviveka॒\,~॥\LARGE\let\thefootnote\relax\footnote{\color{lightgray} Document written using \XeLaTeX{} and chandas font, version 13/11/2016, LSN.}}}
\end{center}
\Large

\centering	
%\raggedright

॥~ādi śaṅkara śloka\,~॥

%\Large
śru̍ti̍ smṛ̍ti pu̍rāṇā॒nām ā̍la̍ya̍ṁ karu̍ṇāla॒ya॒m\,~।\\
na̍mā̍mi̍ bhaga̍vatpā॒daṁ śa̍ṅka̍ra̍ṁ loka̍ śaṅka॒ra॒ṁ\,॥

\vspace{0.5cm}
\Large

rū̍pa̍ṁ dṛ̍śyaṁ loca̍na॒ṁ dṛk ta̍ddṛ̍śya̍ṁ dṛktu̍ māna॒sa॒m\,~।\\
dṛ̍śyā̍ dhī̍vṛtta̍yaḥ sā॒kṣī dṛ̍ge̍va̍ na tu̍ dṛśya॒te॒\,॥\,1\,॥

nī̍la̍pī̍tasthū॒lasū॒kṣma hra̍sva̍dī̍rghādi̍ bheda॒ta॒ḥ॒\,~।\\
nā̍nā̍vi̍dhāni̍ rūpā॒ṇi pa̍śye̍llo̍cana̍meka॒dhā॒\,॥\,2\,॥

ā̍ndhya̍mā̍ndyapa̍ṭutve॒ṣu ne̍tra̍dha̍rmeṣu̍ caika॒dhā॒\,~।\\
sa̍ṁka̍lpa̍yenma̍naḥ śro॒tra tva̍gā̍dau̍ yojya̍tāmi॒da॒m\,॥\,3\,॥

kā̍ma̍ḥ sa̍ṁkalpa̍sande॒hau śra̍ddhā̍'śra̍ddhe dhṛ̍tīta॒re॒\,~।\\
hrī̍rdhī̍rbhī̍ri̍tyeva̍mā॒dīn bhā̍sa̍ya̍tyeka̍dhā ci॒ti॒ḥ॒\,॥\,4\,॥

no̍de̍ti̍ nāsta̍metye॒ṣā na̍ vṛ̍ddhi̍ṁ yāti̍ na kṣa॒ya॒m\,~।\\
sva̍ya̍ṁ vi̍bhātya̍thāna॒nyāni bhā̍sa̍ye̍tsādha̍naṁ vi॒nā॒\,॥\,5\,॥

ci̍cchā̍yā̍'veśa̍to bu॒ddhau bhā̍na̍ṁ dhī̍stu dvi̍dhā sthi॒tā॒\,~।\\
e̍kā̍ha̍ṁkṛti̍ranyā॒ syāt a̍nta̍ḥ ka̍raṇa̍rūpi॒ṇī॒\,॥\,6\,॥

chā̍yā̍'ha̍ṁkāra̍yorai॒kyaṁ ta̍ptā̍ya̍ḥ piṇḍa̍vanma॒ta॒m\,~।\\
ta̍da̍ha̍ṁkāra̍tādā॒tmyāt de̍ha̍śce̍tana̍tāma॒gā॒t॒\,॥\,7\,॥

a̍ha̍ṁkā̍rasya̍ tādā॒tmyaṁ ci̍cchā̍yā̍deha̍sākṣi॒bhi॒ḥ\,~।\\
sa̍ha̍ja̍ṁ karma̍jaṁ bhrā॒nti ja̍nya̍ṁ ca̍ trivi̍dhaṁ kra॒mā॒t\,॥\,8\,॥

\newpage

sa̍ṁba̍ndhi̍noḥ sa̍tornā॒sti ni̍vṛ̍tti̍ḥ saha̍jasya॒ tu॒\,~।\\
ka̍rma̍kṣa̍yāt pra̍bodhā॒cca ni̍va̍rte̍te kra̍mādu॒bhe॒\,॥\,9\,॥

a̍ha̍ṁkā̍rala̍ye su॒ptau bha̍ve̍dde̍ho'pya̍ceta॒na॒ḥ॒\,~।\\
a̍ha̍ṁkā̍rav̍ikāsā॒rdhaḥ sva̍pna̍ḥ sa̍rvastu̍ jāga॒ra॒ḥ॒\,॥\,10\,॥

a̍nta̍ḥ ka̍raṇa̍vṛtti॒śca ci̍ti̍cca̍chā̍yaikya̍māga॒tā॒\,~।\\
vā̍sa̍nā̍ḥ kalpa̍yet sva॒pne bo̍dhe̍'kṣai̍rviśa̍yān ba॒hi॒ḥ॒\,॥\,11\,॥

ma̍no̍'ha̍ṁkṛtyu̍pādā॒naṁ li̍ṅga̍me̍kaṁ ja̍ḍātma॒ka॒m\,~।\\
a̍va̍sthā̍traya̍manve॒ti jā̍ya̍te̍ mriya̍te ta॒thā॒\,॥\,12\,॥


śa̍kti̍dva̍yaṁ hi̍ māyā॒yā vi̍kṣe̍pā̍vṛti̍rūpa॒ka॒m\,~।\\
vi̍kṣe̍pa̍śa̍kti॒rliṅgā॒di bra̍hmā̍ṇḍā̍ntaṁ ja̍gat sṛ॒je॒t॒\,॥\,13\,॥

sṛ̍ṣṭi̍rnā̍ma bra̍hmarū॒pe sa̍cci̍dā̍nanda̍vastu॒ni॒\,~।\\
a̍bdhau̍ phe̍nādi̍vat sa॒rva nā̍ma̍rū̍papra̍sāra॒ṇā॒\,॥\,14\,॥

a̍nta̍rdṛ̍gdṛśya̍yorbhe॒daṁ ba̍hi̍śca̍ brahma̍sarga॒yo॒ḥ॒\,~।\\
ā̍vṛ̍ṇo̍tyapa̍rā śa॒ktiḥ sā̍ sa̍ṁsā̍rasya̍ kāra॒ṇa॒m\,॥\,15\,॥

sā̍kṣi̍na̍ḥ pura̍to bhā॒ti li̍ṅga̍ṁ de̍hena̍ saṁyu॒ta॒m\,~।\\
ci̍ti̍cchā̍yāsa̍māve॒śāt jī̍va̍ḥ syā̍dvyāva̍hāri॒ka॒ḥ॒\,॥\,16\,॥

a̍sya̍ jī̍vatva̍māro॒pāt sā̍kṣi̍ṇya̍pyava̍bhāsa॒te॒\,~।\\
ā̍vṛ̍tau̍ tu vi̍naṣṭā॒yāṁ bhe̍de̍ bhā̍te'pa̍yāti॒ ta॒t॒\,॥\,17\,॥

ta̍thā̍ sa̍rgabra̍hmaṇo॒śca bhe̍da̍mā̍vṛtya̍ tiṣṭha॒ti॒\,~।\\
yā̍ śa̍kti̍stadva̍śādbra॒hma vi̍kṛ̍ta̍tvena̍ bhāsa॒te॒\,॥\,18\,॥

a̍trā̍pyā̍vṛti̍nāśe॒na vi̍bhā̍ti̍ brahma̍sarga॒yo॒ḥ॒\,~।\\
bhe̍da̍sta̍yorvi̍kāra॒ḥ syāt sa̍rge̍ na̍ brahma̍ṇi kva॒ci॒t॒\,॥\,19\,॥

\newpage

a̍sti̍ bhā̍ti pri̍yaṁ rū॒paṁ nā̍ma̍ ce̍tyaṁśa̍pañca॒ka॒m\,~।\\
 ā̍dya̍tra̍yaṁ brahma̍rū॒paṁ  ja̍ga̍drū̍paṁ ta̍to dva॒ya॒m\,॥\,20\,॥

kha̍vā̍yva̍gnija̍lorvī॒ṣu de̍va̍ti̍ryaṅna̍rādi॒ṣu॒\,~।\\
a̍bhi̍nnā̍ḥ sacci̍dāna॒ndāḥ bhi̍dye̍te̍ rūpa̍nāma॒nī॒\,॥\,21\,॥

%%

u̍pe̍kṣya̍ nāma̍rūpe॒ dve sa̍cci̍dā̍nanda̍tatpa॒ra॒ḥ॒\,~।\\
sa̍mā̍dhi̍ṁ sarva̍dā ku॒ryāt hṛ̍da̍ye̍ vā'tha̍vā ba॒hi॒ḥ॒\,॥\,22\,॥

sa̍vi̍ka̍lpo nirvi̍ka॒lpaḥ sa̍mā̍dhi̍rdvivi̍dho hṛ॒di॒\,~।\\
dṛ̍śya̍śa̍bdānu̍vedhe॒na sa̍vi̍ka̍lpaḥ pu̍nardvi॒dhā॒\,॥\,23\,॥

kā̍mā̍dyā̍ścitta̍gā dṛ॒śyāḥ ta̍tsā̍kṣi̍tvena̍ ceta॒na॒m\,~।\\
dhyā̍ye̍ddṛ̍śyānu̍viddho॒'yaṁ sa̍mā̍dhi̍ḥ savi̍kalpa॒ka॒ḥ॒\,॥\,24\,॥

a̍sa̍ṁga̍ḥ sacci̍dāna॒ndaḥ sva̍pra̍bho̍ dvaita̍varji॒ta॒ḥ॒\,~।\\
a̍smī̍ti̍ śabda̍viddho॒'yaṁ sa̍mā̍dhi̍ḥ savi̍kalpa॒ka॒ḥ॒\,॥\,25\,॥

svā̍nu̍bhū̍tira̍sāve॒śāt dṛ̍śya̍śa̍bdāvu̍pekṣya॒ tu॒\,~।\\
ni̍rvi̍ka̍lpaḥ sa̍mādhi॒ḥ syāt ni̍vā̍ta̍sthita̍dīpa॒va॒t॒\,॥\,26\,॥

hṛ̍dī̍va̍ bāhya̍deśe॒'pi ya̍smi̍n ka̍smiṁśca̍ vastu॒ni॒\,~।\\
sa̍mā̍dhi̍rādya̍ḥ sanmā॒trāt nā̍ma̍rū̍papṛ̍thakkṛ॒ti॒ḥ॒\,॥\,27\,॥

a̍kha̍ṇḍai̍kara̍saṁ va॒stu sa̍cci̍dā̍nanda̍lakṣa॒ṇam\,~।\\
i̍tya̍vi̍cchinna̍cinte॒yaṁ sa̍mā̍dhi̍rmadhya̍mo bha॒ve॒t॒\,॥\,28\,॥

sta̍bdhī̍bhā̍vo ra̍sāsvā॒dāt tṛ̍tī̍ya̍ḥ pūrva̍vanma॒ta॒ḥ॒\,~।\\
e̍tai̍ḥ sa̍mādhi̍bhiḥ ṣa॒ḍbhiḥ na̍ye̍t kā̍laṁ ni̍ranta॒ra॒m\,॥\,29\,॥

\newpage

de̍hā̍bhi̍mā̍ne gali॒te vi̍jñā̍te̍ para̍mātma॒ni॒\,~।\\
ya̍tra̍ ya̍tra ma̍no yā॒ti ta̍tra̍ ta̍tra sa̍mādha॒ya॒ḥ\,॥\,30\,॥

bhi̍dya̍te̍ hṛda̍yagra॒nthiḥ chi̍dya̍nte̍ sarva̍saṁśa॒yā॒ḥ\,~।\\
kṣī̍ya̍nte̍ cāsya̍ karmā॒ṇi ta̍smi̍n dṛ̍ṣṭe pa̍rāva॒re॒\,॥\,31\,॥

a̍va̍cchi̍nnaści̍dābhā॒saḥ tṛ̍tī̍ya̍ḥ svapna̍kalpi॒taḥ\,~।\\
vi̍jñe̍ya̍strivi̍dho jī॒vaḥ ta̍trā̍dya̍ḥ pāra̍mārthi॒ka॒ḥ॒\,॥\,32\,॥

a̍va̍cche̍daḥ ka̍lpita॒ḥ syāt a̍va̍cche̍dyaṁ tu̍ vāsta॒va॒m\,~।\\
ta̍smi̍n jī̍vatva̍māro॒pāt bra̍hma̍tva̍ṁ tu sva̍bhāva॒ta॒ḥ॒\,॥\,33\,॥

a̍va̍cchi̍nnasya̍ jīva॒sya pū̍rṇe̍na̍ brahma̍ṇaika॒tā॒m\,~।\\
ta̍ttva̍ma̍syādi̍vākyā॒ni ja̍gu̍rne̍tara̍jīva॒yo॒ḥ॒\,॥\,34॥

bra̍hma̍ṇya̍vasthi̍tā mā॒yā vi̍kṣe̍pā̍vṛti̍rūpi॒ṇī॒\,~।\\
ā̍vṛ̍tya̍kha̍ṇḍa̍tāṁ ta॒smin ja̍ga̍jjī̍vau pra̍kalpa॒ye॒t॒\,॥\,35॥

jī̍vo̍ dhī̍sthaci̍dābhā॒saḥ bha̍ve̍dbho̍ktā hi̍ karma॒kṛ॒t॒\,~।\\
bho̍gya̍rū̍pami̍daṁ sa॒rvaṁ ja̍ga̍t syā̍dbhūta̍bhauti॒ka॒m\,॥\,36\,॥

a̍nā̍di̍kāla̍māra॒bhya mo̍kṣā̍t pū̍rvami̍daṁ dva॒ya॒m\,~।\\
vya̍va̍hā̍re sthi̍taṁ ta॒smāt u̍bha̍ya̍ṁ vyāva̍hāri॒ka॒m\,॥\,37\,॥

ci̍dā̍bhā̍sasthi̍tā ni॒drā vi̍kṣe̍pā̍vṛti̍rūpi॒ṇī॒\,~।\\
ā̍vṛ̍tya̍ jīva̍jaga॒tī pū̍rve̍ nū̍tne tu̍ kalpa॒ye॒t॒\,॥\,38\,॥

pra̍tī̍ti̍kāla॒ evai॒te sthi̍ta̍tvā̍t prāti̍bhāsi॒ke॒\,~।\\
na̍ hi̍ sva̍pnapra॒buddha॒sya pu̍na̍ḥ sva̍pne sthi̍tista॒yo॒ḥ॒\,॥\,39\,॥

prā̍ti̍bhā̍sika̍jīvo॒ yaḥ ta̍jja̍ga̍t prāti̍bhāsi॒ka॒m\,~।\\
vā̍sta̍va̍ṁ manya̍te'nya॒stu mi̍thye̍ti̍ vyāva̍hāri॒ka॒ḥ॒\,॥\,40\,॥

\newpage

vyā̍va̍hā̍rika̍jīvo॒ yaḥ ta̍jja̍ga̍dvyāva̍hāri॒ka॒m\,~।\\
sa̍tya̍ṁ pra̍tyeti̍ mithye॒ti ma̍nya̍te̍ pāra̍mārthi॒ka॒ḥ॒\,॥\,41\,॥

pā̍ra̍mā̍rthika̍jīva॒stu bra̍hmai̍kya̍ṁ pāra̍mārthi॒ka॒m\,~।\\
pra̍tye̍ti vī̍kṣate nā॒nyat vī̍kṣa̍te̍ tvanṛ̍tātma॒nā॒\,॥\,42\,॥

mā̍dhu̍rya̍drava̍śaityā॒ni nī̍ra̍dha̍rmāsta̍raṅga॒ke॒\,~।\\
a̍nu̍ga̍myātha̍ tanni॒ṣṭhe phe̍ne̍'pya̍nuga̍tā ya॒thā॒\,॥\,43\,॥

sā̍kṣi̍sthā̍ḥ sacci̍dāna॒ndāḥ sa̍ṁba̍ndhā̍dvyāva̍hāri॒ke॒\,~।\\
ta̍ddvā̍re̍ṇānu̍gaccha॒nti ta̍thai̍va̍ prāti̍bhāsi॒ke॒\,॥\,44\,॥

la̍ye̍ phe̍nasya̍ taddha॒rmā dra̍vā̍dyā̍ḥ syusta̍raṅga॒ke॒\,~।\\
ta̍syā̍pi̍ vila̍ye nī॒re ti̍ṣṭha̍ntye̍te ya̍thā pu॒rā॒\,॥\,45\,॥

prā̍ti̍bhā̍sika̍jīva॒sya la̍ye̍ syu̍rvyāva̍hāri॒ke\,~।\\
ta̍lla̍ye̍ sacci̍dāna॒ndāḥ pa̍rya̍va̍syanti̍ sākṣi॒ṇi॒\,॥\,46\,॥

॥~\,i॒ti bhā॒ratī tīrtha svāminā viracitaḥ
dṛgdṛśyavivekaḥ samā॒pta॒ḥ॒\,~॥ \\

\vspace{2cm}


\end{document}