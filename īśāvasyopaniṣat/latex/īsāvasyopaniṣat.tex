%!TEX encoding = UTF-8 Unicode
%!TEX TS-program = xelatex

%\documentclass[11pt, a4paper]{article}
\documentclass[parskip, DIV=14]{scrartcl}
\usepackage{fontspec}
\usepackage{xltxtra} % a few fixes and extras
\usepackage{fancyhdr} % header

\usepackage{xcolor} % writing in color

\renewcommand{\baselinestretch}{1.2}% interligne + 20%

%\setmainfont[Ligatures=TeX]{Linux Libertine}
%\setmainfont[Ligatures=TeX]{Sanskrit 2003}
%\setmainfont[Ligatures=TeX]{uttara}
\setmainfont[Ligatures=TeX]{Chandas}

% footnote in end of document
\interfootnotelinepenalty=10000

\renewcommand{\thefootnote}{\fnsymbol{footnote}} %A sequence of nine symbols, try it and see!


% add a small title: name of the file + initials
\pagestyle{fancy}
\lhead{\color{lightgray} īśāvāsyopaniṣat}
%\lhead{\color{lightgray} īśopaniṣat}
\rhead{\color{lightgray} LSN}

\setkomafont{pagenumber}{\normalfont\bfseries}
\pagenumbering{roman}
\begin{document}

%\pagenumbering{gobble} % remove page numebr

\vspace{-1.5cm}

\begin{center}
% see command for footmark without numbering
\textbf{{\Huge॥\,~īśāvāsyopaniṣat\,~॥  \LARGE\let\thefootnote\relax\footnote{\color{lightgray} Document written using \XeLaTeX{} and chandas font,  version 04/05/2019, LSN.}}}
\end{center}
\Large

% only  Upanisad that forms an integral part of a vedic Samhita
%Notation of the śukla yajurveda
% cf śrī sūktam

% two recensions:  the Madhyandina and the Kanva

% īśa: the Creator, the Supreme Intelligence

% mantropaniṣad, since it come from the end of the saṁhita of the śukla yajurveda
% Most upaniṣad comes from the brahmaṇa part of the veda and are called brahmopaniṣad

{\centering	
%\raggedright

%%%%%%%%%%%%%%%%%% Gḥāṇā %%%%%%%%%%%%%%%%%%%%%

\large
14th chapter of the vājasaneyi saṁhitā of the śukla yajurveda\\
Other names:  saṁhita upaniṣat, īśopaniṣat, vājasaneyi saṁhitā upaniṣad

\Large
॥\,~śānti pāṭhaḥ\,~॥\\
\vspace{0.5cm}

oṁ pūrṇa॒mada॒ḥ pūrṇa॒mida॒ṁ pūrṇā॒tpūrṇa॒muda॒cyate~।
pūrṇa॒sya pūrṇa॒mādā॒ya pūrṇa॒mevāvaśi॒ṣyate~॥

oṁ śā॒ntiḥ śā॒ntiḥ śā॒ntiḥ~॥

॥\,~atheśāvāsyopaniṣadadi \,~॥\\
॥\,~atha īśopaniṣat \,~॥\\
\vspace{0.5cm}

om ī॒śā vā॒sya̍mi॒dagṁ sarva॒ṁ yat kiṁ ca॒ jaga̍tyā॒ṁ jaga̍t~।
tena̍ tya॒ktena̍ bhuñjīthā॒ mā gṛ̍dha॒ḥ kasya̍svi॒ddhanam̎~॥~1~॥

ku॒rvanne॒veha karmā̎ṇi jijīvi॒ṣeccha॒tagṁ samā̎ḥ~।
e॒vaṁ tvayi॒ nānyathe॒to̎'sti॒ na karma̍ lipyate॒ nare̎~॥~2~॥

a॒su॒ryā॒ nāma॒ te lo॒kā a॒ndhena॒ tama॒sā''vṛ̍tāḥ~।
tāggste pretyā॒bhiga̍cchanti॒ ye ke cā̎tma॒hano॒ janā̎ḥ~॥~3~॥

ane̎ja॒deka॒ṁ mana̍so॒ javī̎yo॒ naina̍dde॒vā ā̎pnuva॒npūrva॒marṣa̍t~।
taddhāva̍to॒'nyānatye̎ti॒ tiṣṭha॒ttasmin̎na॒po mā̎ta॒riśvā̎ dadhāti~॥~4~॥

tade̎jati॒ tanneja̍ti॒ taddū॒re tadvan̎ti॒ke~।
tada॒ntara̍sya॒ sarva̍sya॒ tadu॒ sarva̍syāsya bāhya॒taḥ~॥~5~॥

yastu sarvā̎ṇi bhū॒tānyā॒tmanye॒vānu॒paśya̍ti~।
sa॒rva॒bhū॒teṣu̍ cā॒tmāna॒ṁ tato॒ na viju̍gupsate~॥~6~॥

yasmi॒nsarvā̎ṇi bhū॒tānyā॒tmaivābhū̎dvijāna॒taḥ~।
tatra॒ ko moha॒ḥ kaḥ śoka̍ eka॒tvama̍nu॒paśya̍taḥ~॥~7~॥

sa parya̍gācchu॒krama̍kā॒yama̍vra॒ṇama̍snāvi॒ragṁ śu॒ddhamapā̎paviddham~।
ka॒virma̍nī॒ṣī pa̍ri॒bhūḥ sva̍ya॒ṁbhūryā̎thātathya॒to'rthā॒n vya̍dadhācchāśva॒tībhya॒ḥ samā̎bhyaḥ~॥~8~॥

a॒ndhaṁ tama॒ḥ pravi̍śanti॒ ye'vi̍dyāmu॒pāsa̍te~।
tato॒ bhūya̍ iva॒ te tamo॒ ya u̍ vi॒dyāyā̎gṁ ra॒tāḥ~॥~9~॥

a॒nyade॒vāhurvi॒dyayā॒'nyadā̎hu॒ravi̍dyayā~।
iti̍ śuśruma॒ dhīrā̎ṇā॒ṁ ye na॒stadvi̍cacakṣi॒re~॥~10~॥

vi॒dyāṁ cāvi̍dyāṁ ca॒ yastadvedo॒bhaya̍gṁ sa॒ha~।
avi̍dyayā mṛ॒tyuṁ tī॒rtvā vi॒dyayā॒'mṛta̍maśnute~॥~11~॥

%APRIL 13
%vidya 
%More vidya you have more humble yo need to be  (other avidya will easy take overt egoist and arrogance)

a॒ndhaṁ tama॒ḥ pravi̍śanti॒ ye'sam̎bhūtimu॒pāsa̍te~।
tato॒ bhūya̍ iva॒ te tamo॒ ya u॒ sambhū̎tyāgṁ ra॒tāḥ~॥~12~॥

a॒nyade॒vāhuḥ sam̎bha॒vāda॒nyadā̎hu॒rasam̎bhavāt~।
iti̍ śuśruma॒ dhīrā̎ṇā॒ṁ ye na॒stadvi̍cacakṣi॒re~॥~13~॥

sambhū̎tiṁ ca vinā॒śaṁ ca॒ yastadvedo॒bhaya̍gṁ sa॒ha~।
vi॒nā॒śena̍ mṛ॒tyuṁ tī॒rtvā sambhū̎tyā॒'mṛta̍maśnute~॥~14~॥

hi॒ra॒ṇmaye̎na॒ pātre̎ṇa sa॒tyasyāpi̍hita॒ṁ mukham̎~।
tattvaṁ pū̎ṣa॒nnapāvṛ̍ṇu sa॒tyadha̎rmāya dṛ॒ṣṭaye̎~॥~15~॥

pūṣa̍nnekarṣe yama sūrya॒ prājā̎patya॒ vyū̎ha ra॒śmīn samū̎ha॒
tejo॒ yatte̎ rū॒paṁ kalyā̎ṇatama॒ṁ tatte̎ paśyāmi~।
yo॒'sāva॒sau puru̍ṣa॒ḥ so॒'hama̍smi~॥~16~॥

vā॒yurani̍lama॒mṛta॒mathe॒daṁ bhasmā̎nta॒gṁ॒ śarī̍ram~।
oṁ krato॒ smara̍ kṛ॒tagṁ sma̍ra॒ krato॒ smara̍ kṛ॒tagṁ sma̍ra~॥~17~॥

agne॒ naya̍ su॒pathā̎ rā॒ye a॒smān viśvā̍ni deva va॒yunā̍ni vi॒dvān~।
yu॒yo॒dhya॒smajju̍hurā॒ṇameno॒ bhūyi̍ṣṭhāṁ te॒ nama̍ uktiṁ vidhema~॥~18~॥

॥\,~iti īśopaniṣat \,~॥\\

oṁ pūrṇa॒mada॒ḥ pūrṇa॒mida॒ṁ pūrṇā॒tpūrṇa॒muda॒cyate~।
pūrṇa॒sya pūrṇa॒mādā॒ya pūrṇa॒mevāvaśi॒ṣyate~॥

oṁ śā॒ntiḥ śā॒ntiḥ śā॒ntiḥ~॥

\Large
%%%%%%%%%%%%%%%%%%%%%%%%%%%%%%%%%%%%%%%%%%
% visible / unvisible 

\end{document}