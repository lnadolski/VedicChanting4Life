%!TEX encoding = UTF-8 Unicode
%!TEX TS-program = xelatex

%\documentclass[11pt, a4paper]{article}
\documentclass[parskip, DIV=14]{scrartcl}
\usepackage{fontspec}
\usepackage{xltxtra} % a few fixes and extras

\usepackage{xcolor} % writing in color


%\setmainfont[Ligatures=ṭeX]{Linux Libertine}
%\setmainfont[Ligatures=ṭeX]{Sanskrit 2003}
\setmainfont[Ligatures=TeX]{Chandas}

% footnote in end of document
\interfootnotelinepenalty=10000

\renewcommand{\thefootnote}{\fnsymbol{footnote}} %A sequence of nine symbols, try it and see!

\begin{document}

%\pagenumbering{gobble} % remove page numebr

\vspace{-1.5cm}

\begin{center}
% see command for footmark without numbering
\textbf{{\Huge\vspace{0.201cm} ॥  śrī gā॒yatrīmantrāḥ॒ ॥\LARGE\let\thefootnote\relax\footnote{\color{lightgray} Document written using \XeLaTeX{} and chandas font, version 22/04/2015, LSN.}}}
\end{center}
\Large

%\centering	
%\raggedright

\vspace{1.001cm} 

\LARGE Other gāyatrīmantrāḥ from mahānārāyaṇa upaniṣat (drāviḍapāṭha recession) \par

\Large 

\vspace{1.501cm} ॥   sūrya  ॥ \par
  om  । bhā॒ska॒rāya̍ vi॒dmahe̍ mahaddyutika॒rāya̍ dhīmahi  ।  tanna̍ ādityaḥ praco॒dayā̎t ॥ \\


\vspace{0.201cm} ॥   nandī  ॥ \par
  om  । tatpuru̍ṣāya vi॒dmahe̍ cakratu॒ṇḍāya̍ dhīmahi  ।  tanno̍ nandī praco॒dayā̎t ॥ \\

\vspace{0.201cm} ॥   ṣaṇmukha  ॥ \par
  om  । tatpuru̍ṣāya vi॒dmahe̍ mahāse॒nāya̍ dhīmahi  ।  tanna̍ḥ ṣaṇmukhaḥ praco॒dayā̎t ॥ \\

\vspace{0.201cm} ॥   brahmā  ॥ \par
  om  । ve॒dā॒tma॒nāya̍ vi॒dmahe̍ hiraṇyaga॒rbhāya̍ dhīmahi  ।  tanno̎ brahmā praco॒dayā̎t ॥ \\

\vspace{0.201cm} ॥   nṛsiṁha  ॥ \par  
  om  । va॒jra॒na॒khāya̍ vi॒dmahe̍ tīkṣṇada॒ṁṣṭrāya̍ dhīmahi  ।  tanno̍ nārasiṁhaḥ praco॒dayā̎t ॥ \\

\vspace{0.201cm} ॥   agni  ॥ \par
  om  । vai॒śvā॒na॒rāya̍ vi॒dmahe̍ lālī॒lāya̍ dhīmahi  ।  tanno̍ agniḥ praco॒dayā̎t ॥ \\

\newpage
\LARGE  Other gāyatrīmantrāḥ (dīpakā of nārāyaṇa)

\Large 
\vspace{1.501cm} ॥   brahmā  ॥ \par
  om  । ca॒tu॒rmu॒khāya̍ vi॒dmahe̍ kamaṇḍaludha॒rāya̍ dhīmahi  ।  tanno̎ brahmā praco॒dayā̎t ॥ \\

\vspace{0.201cm} ॥   sūrya  ॥ \par
  om  । ā॒di॒tyāya̍ vi॒dmahe̍ sahasrakira॒ṇāya̍ dhīmahi  ।  tanno̍ bhānuḥ praco॒dayā̎t ॥ \\

\vspace{0.201cm} ॥   vaiśvānara ॥ \par
  om  । pāva̍kāya vi॒dmahe̍ saptaji॒hvāya̍ dhīmahi  ।  tanno̍ vaiśvānaraḥ praco॒dayā̎t ॥ \\

\vspace{0.201cm} ॥   durgā  ॥ \par
  om  । ma॒hā॒śū॒linyai̍ vi॒dmahe̍ mahādu॒rgāyai̍ dhīmahi  ।  tanno̍ bhagavatī praco॒dayā̎t ॥ \\

\vspace{0.201cm} ॥   gaurī  ॥ \par
  om  । su॒bha॒gā॒yai ca̍ vi॒dmahe̍ kāmalamāli॒nyai ca̍ dhīmahi  ।  tanno̍ gaurī praco॒dayā̎t ॥ \\

\vspace{0.201cm} ॥   sarpa ॥ \par
  om  । na॒va॒ku॒lāya̍ vi॒dmahe̍ viṣada॒ntāya̍ dhīmahi  ।  tanna̍ḥ sarpaḥ praco॒dayā̎t ॥ \\

\vspace{0.201cm} ॥   lakṣmī  ॥ \par
  om  । ma॒hā॒de॒vyai ca̍ vi॒dmahe̍ viṣṇupa॒tnyai ca̍ dhīmahi   ।  tanno̍ lakṣmīḥ praco॒dayā̎t ॥ \\

\vspace{0.201cm} ॥   rāma  ॥ \par
  om  । ra॒ghu॒va॒ṁśyāya̍ vi॒dmahe̍ sītāvalla॒bhāya̍ dhīmahi  ।  tanno̍ rāmaḥ praco॒dayā̎t ॥ \\
  om  । dā॒śa॒ra॒thāya̍ vi॒dmahe̍ sītāvalla॒bhāya̍ dhīmahi  ।  tanno̍ rāmaḥ praco॒dayā̎t ॥ \\

\vspace{0.201cm} ॥   śaṅkara  ॥ \par
  om  । sa॒dā॒śi॒vāya̍ vi॒dmahe̍ sahasrā॒kṣyāya̍ dhīmahi  ।  tanna̍ḥ sāmbaḥ praco॒dayā̎t ॥ \\

\vspace{0.201cm} ॥   kālī  ॥ \par
  om  । kā॒li॒kā॒yai ca̍ vi॒dmahe̍ śmaśānavāsi॒nyai ca̍ dhīmahi  ।  tanna̍ aghorā praco॒dayā̎t ॥ \\

\end{document}