%!TEX encoding = UTF-8 Unicode
%!TEX TS-program = xelatex

%\documentclass[11pt, a4paper]{article}
\documentclass[parskip, DIV=18]{scrartcl}
\usepackage{fontspec}
\usepackage{xltxtra} % a few fixes and extras

\usepackage{xcolor} % writing in color


%\setmainfont[Ligatures=TeX]{Linux Libertine}
%\setmainfont[Ligatures=TeX]{Sanskrit 2003}
\setmainfont[Ligatures=TeX]{Chandas}

% Requires font Nakula by John Smith,
% see http://bombay.indology.info/software/fonts/devanagari/indexḥtml
\newcommand\skt{\catcode`\~=12
           \fontspec[Script=Devanagari,Mapping=velthuis-sanskrit]{Chandas}}
\newfontfamily
  \sanskritfont [Script=Devanagari,Mapping=velthuis-sanskrit]{Sanskrit 2003}
%  \sanskritfont [Script=Devanagari,Mapping=velthuis-sanskrit]{Sahadeva}
%\sanskritfont [Script=Devanagari,Mapping=velthuis-sanskrit]{Devanagari Sangam MN}

\renewcommand{\thefootnote}{\fnsymbol{footnote}} %A sequence of nine symbols, try it and see!

\begin{document}


\pagenumbering{gobble} % remove page numebr

\vspace{-1.5cm}

\begin{center}
\textbf{{\Huge śrī a॒ṣṭākṣarī sthutiḥ॒} \LARGE\let\thefootnote\relax\footnote{ \color{gray} Document written using \XeLaTeX{} and chandas font, LSN, version 27/12/2014.}}
\end{center}
\Large

\centering
%\raggedright

\vspace{0.5cm}

oṁ na̍maḥ praṇa̍vārthā॒rtha sthū̍la̍sū̍kṣma kṣa̍rākṣa॒ra॒\,। \\
vya॒ktā̍vya̍kta ka̍lātī॒ta oṁ̍kā̍rā̍ya na̍mo na॒ma॒ḥ॒\,॥\,1\,॥ \par 

%%\vspace{0.5cm}

\noindent na̍mo devādi॑devā॒ya de॑ha̍sa̍ñcāra̍heta॒ve॒\,। \\
dā̍tya̍saṅgha vi॑nāśā॒ya na̍kā̍rā̍ya na̍mo na॒ma॒ḥ॒\,॥\,2\,॥ \par

%\vspace{0.5cm}

\noindent mo̍hanaṁ viśva̍rūpa॒ṁ ca śi॒ṣṭā̍cā̍ra su॑poṣi॒ta॒m\,। \\
mo̍ha vidhvaṁsa̍kaṁ va॒nde mo̍kā̍rā̍ya na̍mo na॒ma॒ḥ॒\,॥\,3\,॥ \par

%\vspace{0.5cm}

\noindent nā̍rāyaṇāya̍navyā॒ya na̍ra̍si॑ṁhāya̍nāmi॒ne॒\,। \\
nā̍dāya nādi॑ne tu॒bhyaṁ nā̍kā̍rā̍ya na̍mo na॒ma॒ḥ॒\,॥\,4\,॥ \par

%\vspace{0.5cm}

\noindent rā̍macandraṁ ra̍ghupa॒tiṁ pi॒trā̍jñya pari॑pāla॒ka॒m\,। \\
kā̍salyātana̍yaṁ va॒nde rā̍kā̍rā̍ya na̍mo na॒ma॒ḥ॒\,॥\,5\,॥ \par

%\vspace{0.5cm}

\noindent ya̍jñā̍ya yajña̍gamyā॒ya ya̍jña̍ra̍kṣaka̍rāya॒ ca॒\,। \\
ya॒jñā̍ṅga rūpi॑ṇe tu॒bhyaṁ ya̍kā̍rā̍ya na̍mo na॒ma॒ḥ॒\,॥\,6\,॥ \par

%\vspace{0.5cm}

\noindent ṇā̍kāraṁ loka̍vikhyā॒taṁ nā̍nā̍ja̍nma pha̍lapra॒da॒m\,। \\
nā̍nābhīṣṭapra̍daṁ va॒nde ṇā̍kā̍rā̍ya na̍mo na॒ma॒ḥ॒\,॥\,7\,॥ \par

%\vspace{0.5cm}

\noindent ya̍jñakartre ya̍jñabha॒rtre ya̍jña̍rū̍pāya̍te na॒ma॒ḥ॒\,। \\
su॑jñāna goca̍rāyā॒stu ya̍kā̍rā̍ya na̍mo na॒ma॒ḥ॒\,॥\,8\,॥ \par

%\vspace{0.5cm}

\noindent oṁ̍kāramantra̍saṁyu॒ktaṁ ni॑tya̍ṁ dhyā̍yanti॑ yogi॒na॒ḥ॒\,। \\
kā̍madaṁ mokṣa̍daṁ ta॒smai o̍ṁkā̍rā̍ya na̍mo na॒ma॒ḥ॒\,॥\,9\,॥ \par

%\vspace{0.5cm}

\noindent nā̍rāyaṇaḥ pa̍raṁ bra॒hma nā̍rā̍yaṇaḥ pa̍ranta॒pa॒ḥ॒\,। \\
nā̍rāyaṇaḥ pa̍ro de॒vaḥ sa̍rva̍ṁ nā̍rāya̍ṇassa॒dā॒\,॥\,10\,॥ \par
 
\vspace{0.5cm}
 
\begin{center}
 ॥\,iti a॒ṣṭākṣarī stu॒ti॒ḥ saṁpū॒rṇā\,॥
\end{center}

\vspace{1.5cm}

\end{document}