%!TEX encoding = UTF-8 Unicode
%!TEX TS-program = xelatex

%\documentclass[11pt, a4paper]{article}
\documentclass[parskip, DIV=14, pagenumber=head,top]{scrartcl}
\usepackage{fontspec}
\usepackage{xltxtra} % a few fixes and extras
\usepackage{fancyhdr} % header


\usepackage{xcolor} % writing in color

\renewcommand{\baselinestretch}{1.2}% interligne + 20%

%\setmainfont[Ligatures=TeX]{Linux Libertine}
%\setmainfont[Ligatures=TeX]{Sanskrit 2003}
\setmainfont[Ligatures=TeX]{Chandas}

% footnote in end of document
\interfootnotelinepenalty=10000

\renewcommand{\thefootnote}{\fnsymbol{footnote}} %A sequence of nine symbols, try it and see!

% add a small title: name of the file + initials
\pagestyle{fancy}
\lhead{\color{lightgray} yogāñjalisāram (extraits)}
\rhead{\color{lightgray} LSN}

\setkomafont{pagenumber}{\normalfont\bfseries}
\pagenumbering{roman}


\begin{document}

%\pagenumbering{gobble} % remove page numebr

\vspace{-1.5cm}

\begin{center}
% see command forfootmark without numbering
\textbf{{\Huge॥\,~yogāñjalisāram (extraits)\,~॥\\
\vspace{0.5cm}
\LARGE T.~Krishnamacharya
\LARGE\let\thefootnote\relax\footnote{\color{lightgray} Document written using \XeLaTeX{} and chandas font, version 10/12/2018, LSN.}}}
\end{center}
\Large

\centering	
%\raggedright

ra̍kṣa prathamaṁ ca̍kṣuḥ śrotraṁ\\
nā̍sāṁ jihvāṁ ta॒da̍nu̍tvāṁ ca\,~।\\
hṛ॒da̍ya̍ṁ tundaṁ nā̍bhiṁ yoniṁ\\
ta॒ta̍stu rakṣet sa॒ka̍laṁ gātram\,~॥3॥

dṛ̍ṣṭvā smṛtvā spṛ̍ṣṭvā viṣayaṁ\\
mo̍haṁ mā kuru ma॒na̍si manuṣya\,~।\\
jñā̍tvā sarvaṁ bā̍hyamanityaṁ\\
ni̍ścinu nityaṁ pṛ॒tha̍gātmānam\,~॥5॥

nṛ̍tyati yogī hṛ॒da̍ye dhṛtvā\\
su̍ndaravapuṣaṁ la̍kṣmīkāntam\,~।\\
ja॒ga̍dā̍dhāraṁ pa॒ra̍mā̍tmānaṁ\\
na̍ndati na̍ndati na̍ndatyeva\,~॥8॥

ā̍tmika॒ dai̍hika॒ mā̍nasa bhedāt\\
tri॒vi̍dhaṁ vihitaṁ yo̍gābhyasanam\,~।\\
sa॒ka̍la̍ṁ yacchati vā̍ñchita suphalaṁ\\
na॒hi nahi yogābhya॒sanaṁ viphalam\,~॥11॥

a̍ṣṭāṅgākhyaṁ yo̍gābhyasanaṁ\\
mu̍ktiṁ bhuktiṁ pra॒da̍dā̍tyanaghām\,~।\\
ya̍di guru padavīma॒nu̍gatamatha vā\\
ci̍ttaṁ bhagavatpa॒da̍yo̍rlagnam\,~॥12॥

\newpage 
rā̍go rogotpa̍ttau bījaṁ\\
bho̍go rogapra॒sa̍raṇa bījam\,~।\\ 
yo̍go rāgacche̍dakabījaṁ\\
yā̍hi sudūraṁ rā̍gātbhogāt\,~॥16॥

ni̍tyābhyasanāt ni̍ścalabuddhiḥ\\
sa̍ta̍tā̍dhyayanāt me̍dhāsphūrtiḥ\,~।\\
śu̍ddhāddhyānāt a॒bhī̍ṣṭasiddhiḥ\\
sa̍ntata japataḥ sva॒rū̍pasiddhiḥ\,~॥21॥

ā̍sanakaraṇātta॒ra̍saṁ sarasaṁ\\
prā̍ṇāyāmāt pra॒ba̍la̍ṁ prāṇam\,~।\\
dhā̍raṇaśuddhaṁ kuru mastiṣkaṁ\\
dhyā̍nāt śuddhaṁ ci̍ttaṁ nityam\,~॥27॥

\end{document}