%!TEX encoding = UTF-8 Unicode
%!TEX TS-program = xelatex

%\documentclass[11pt, a4paper]{article}
\documentclass[parskip, DIV=14, pagenumber=head,top]{scrartcl}
\usepackage{fontspec}
\usepackage{xltxtra} % a few fixes and extras
\usepackage{fancyhdr} % header


\usepackage{xcolor} % writing in color

\renewcommand{\baselinestretch}{1.2}% interligne + 20%

%\setmainfont[Ligatures=TeX]{Linux Libertine}
%\setmainfont[Ligatures=TeX]{Sanskrit 2003}
\setmainfont[Ligatures=TeX]{Chandas}

% footnote in end of document
\interfootnotelinepenalty=10000

\renewcommand{\thefootnote}{\fnsymbol{footnote}} %A sequence of nine symbols, try it and see!

% add a small title: name of the file + initials
\pagestyle{fancy}
\lhead{\color{lightgray} yogāñjalisāram}
\rhead{\color{lightgray} LSN}

\setkomafont{pagenumber}{\normalfont\bfseries}
\pagenumbering{roman}


\begin{document}

%\pagenumbering{gobble} % remove page numebr

\vspace{-1.5cm}

\begin{center}
% see command forfootmark without numbering
\textbf{{\Huge॥\,~yogāñjalisāram\,~॥\\
\vspace{0.5cm}
\LARGE T.~Krishnamacharya
\LARGE\let\thefootnote\relax\footnote{\color{lightgray} Document written using \XeLaTeX{} and chandas font, version 10/12/2018, LSN.}}}
\end{center}
\Large

\centering	
%\raggedright

gṛ̍ṇu gopālaṁ sma̍ra turagāsyaṁ\\
bha̍ja guruvaryaṁ mandamate\,~।\\
śu̍ṣke rakte kṣī̍ṇe dehe\\
na॒hi̍ nahi rakṣati ka॒li̍yuga śikṣā\,~॥1॥

pi̍ba yogāñja̍lisāraṁ nityaṁ\\
vi̍śa yogāsanama॒mṛ̍ta̍ṁ geham\,~।\\
sthā̍paya vāyuṁ prā̍ṇāyāmāt\\
hṛ॒da̍ye̍ sudṛḍhaṁ sa॒da̍ya̍ṁ satatam\,~॥2॥

ra̍kṣa prathamaṁ ca̍kṣuḥ śrotraṁ\\
nā̍sāṁ jihvāṁ ta॒da̍nu̍tvāṁ ca\,~।\\
hṛ॒da̍ya̍ṁ tundaṁ nā̍bhiṁ yoniṁ\\
ta॒ta̍stu rakṣet sa॒ka̍laṁ gātram\,~॥3॥

mā̍svapa॒ mā̍svapa॒ ka̍lye samaye\\
mā̍ kuru lāpaṁ pi॒śu̍naiḥ puruṣaiḥ\,~।\\
sa̍ṁsmara nityaṁ ha॒ri̍mabjākṣaṁ\\
stu॒hi̍ savitāraṁ su॒va̍rṇa̍varṇam\,~॥4॥

dṛ̍ṣṭvā smṛtvā spṛ̍ṣṭvā viṣayaṁ\\
mo̍haṁ mā kuru ma॒na̍si manuṣya\,~।\\
jñā̍tvā sarvaṁ bā̍hyamanityaṁ\\
ni̍ścinu nityaṁ pṛ॒tha̍gātmānam\,~॥5॥

\newpage 

jñā̍te tatve ka̍ste mohaḥ\\
ci̍tte śuddhe kva॒bha̍ve̍drogaḥ\,~।\\
ba̍ddhe prāṇe kva॒vā̍sti maraṇaṁ\\
ta̍smādyogaḥ śa॒ra̍ṇaṁ bharaṇam\,~॥6॥
nā̍ḍīgranthiṣu ja॒na̍na̍ṁ labdhvā \\
mā̍ṁse kośe vṛ̍ddhiṁ gatvā\,~।\\ 
sa̍ndhiṣu līlāna॒ṭa̍na̍ṁ kṛtvā\\
ro̍go yogānna̍śyati hā hā\,~॥7॥	

nṛ̍tyati yogī hṛ॒da̍ye dhṛtvā\\
su̍ndaravapuṣaṁ la̍kṣmīkāntam\,~।\\
ja॒ga̍dā̍dhāraṁ pa॒ra̍mā̍tmānaṁ\\
na̍ndati na̍ndati na̍ndatyeva\,~॥8॥

ye̍nādhītā śrau̍tī vāṇī\\
nai̍vakadācit su॒kṛ̍tā sandhyā\,~।\\
sa̍ tu vasudhā jīvanabhāgyaṁ\\
dha̍rmaṁ nindati ni̍ndatyeva\,~॥9॥

rā̍go bhogo yo̍gastyāgaḥ\\
ca̍tvāraste pu॒ru̍ṣā̍rthā hi\,~।\\
bā̍lastaruṇo vṛ̍ddho jīrṇaḥ\\
ca̍tvā̍rastān ba॒hu̍manyante\,~॥10॥

ā̍tmika॒ dai̍hika॒ mā̍nasa bhedāt\\
tri॒vi̍dhaṁ vihitaṁ yo̍gābhyasanam\,~।\\
sa॒ka̍la̍ṁ yacchati vā̍ñchita suphalaṁ\\
na॒hi nahi yogābhya॒sanaṁ viphalam\,~॥11॥

\newpage 
a̍ṣṭāṅgākhyaṁ yo̍gābhyasanaṁ\\
mu̍ktiṁ bhuktiṁ pra॒da̍dā̍tyanaghām\,~।\\
ya̍di guru padavīma॒nu̍gatamatha vā\\
ci̍ttaṁ bhagavatpa॒da̍yo̍rlagnam\,~॥12॥

ta̍va̍ vā mama vā sa॒dā̍nusaraṇāt\\
na॒ma̍nā̍nmananāt pra॒sa̍nna cittaḥ\,~।\\
bha॒ga̍va̍n vāñcchitama॒kh̍ila̍ṁ datvā\\
ki̍nte bhūyaḥ pri॒ya̍miti hasati\,~॥13॥

ka̍ste bhrātā kā̍ vā bhāryā\\
ka̍ste mitraḥ ko̍'yaṁ putraḥ\,~।\\
vi̍tte naṣṭe jī̍rṇe gātre\\
dra॒va̍nti sarve vi॒di̍śo dhikdhik\,~॥14॥

yā̍vadvittāṁ tā̍vad bandhuḥ\\
yā̍vaddānaṁ tā̍vatkīrtiḥ\,~।\\
vi̍tte lupte ba̍ndhurdūre\\
kī̍rtiḥ kva syātpa̍śya vicitram\,~॥15॥

rā̍go rogotpa̍ttau bījaṁ\\
bho̍go rogapra॒sa̍raṇa bījam\,~।\\ 
yo̍go rāgacche̍dakabījaṁ\\
yā̍hi sudūraṁ rā̍gātbhogāt\,~॥16॥

tya॒ja̍ dhi̍kkāraṁ mā̍tāpitroḥ\\
ku॒ru̍ nya̍kkāraṁ pi॒śu̍ne̍ manuje\,~।\\
bha॒ja̍ sa̍tkāraṁ bhā̍vuka boddhari\\
va॒sa̍ sa̍dgoṣṭhīva॒sa̍tau̍ satatam\,~॥17॥

\newpage 

mā̍ kuru ṛṇamapya̍lpaṁ heyaṁ\\
mā̍ vasa ripupa॒ri̍vā̍re satatam\,~।\\
mā̍kṣipa rogajva॒la̍ne gātraṁ\\
mā̍ vismara mā̍ramaṇaṁ hṛdaye\,~॥18॥
jñā̍naratovā ka̍rmaratovā\\
bha̍ktiratovā sa̍rve lokāḥ\,~।\\
sthi̍tvā yoge na॒hi̍ nahi labhate\\
kā̍mapi siddhaṁ pa̍śya vicitram\,~॥19॥

ā̍dau pādau ta॒da̍nu̍ ca jaṅghe\\
pa̍ścā̍dūru nā̍bhiṁ hṛdayam\,~।\\
dhyā̍tvā bāhū su̍ndaravapuṣaṁ\\
su॒mu̍kha̍ṁ lokaya॒ go̍kulanātham\,~॥20॥

ni̍tyābhyasanāt ni̍ścalabuddhiḥ\\
sa̍ta̍tā̍dhyayanāt me̍dhāsphūrtiḥ\,~।\\
śu̍ddhāddhyānāt a॒bhī̍ṣṭasiddhiḥ\\
sa̍ntata japataḥ sva॒rū̍pasiddhiḥ\,~॥21॥

dyu̍ma̍ṇe̍rudayāt prā̍gevāsana\\
sa̍ndhyāpūjana vi॒dha̍yaḥ kāryāḥ\,~।\\
yā̍me yāme prā̍ṇāyāmāt\\
da॒śa̍ daśa kuryāt ā̍yurvṛddhyai\,~॥22॥

pa̍ri̍mita bhojī su॒ca̍rita yājī\\
dhva̍staśarīrakle̍śo yogī\,~।\\
su̍sthiracitto bha॒ga̍vati viṣṇau\\
i॒hai̍va labhate śā̍ntiṁ paramām\,~॥23॥

\newpage 

ā̍dāvāsanapu॒na̍rāvṛtteḥ\\
ā̍dyāvṛtterbha॒ga̍vaccaraṇau\,~।\\
gu̍ru̍va̍racaraṇau pra॒ṇa̍mya paścāt\\
sa॒ma̍dṛkprāṇaḥ sa॒mā̍rabheta\,~॥24|

yā̍vān dīrghaḥ kau̍kṣyo vāyuḥ\\
pra॒yā̍ti bāhyaṁ sū̍kṣmastadanu\,~।\\
tā̍vānantaḥ pra॒vi̍śati no vā\\
ma̍tvā̍ manasā sa॒mī̍kuruṣva\,~॥25॥

va̍da vada satyaṁ va॒ca̍naṁ madhuraṁ\\
lo̍kaya lokaṁ sne̍hasupūrṇam\,~।\\
mā̍rjaya doṣān de̍haprabhavān\\
ā̍rjaya vidyāvi॒na̍ya̍dhanāni\,~॥26॥

ā̍sanakaraṇātta॒ra̍saṁ sarasaṁ\\
prā̍ṇāyāmāt pra॒ba̍la̍ṁ prāṇam\,~।\\
dhā̍raṇaśuddhaṁ kuru mastiṣkaṁ\\
dhyā̍nāt śuddhaṁ ci̍ttaṁ nityam\,~॥27॥

kṛ॒te̍ jñāna mārgaḥ tri॒te̍ karma mārgo\\
dva॒ya̍ṁ dvāpare su̍praśastaṁ phalāya\,~।\\
ka॒lau̍ yoga mārgaḥ sadā\\
su॒praśastassu॒bhu̍ktau vimuktau\,~॥28॥

mu॒ni̍rbhuṅkṣva bhojyaṁ sa॒dā̍ deva śeṣaṁ\\
mi॒ta̍ṁ sātvikaṁca̍rdhakālesupakvam\,~।\\
sma॒ra̍n devanāthaṁ ku॒ru̍ṣvārdha pūrṇaṁ\\
sva॒ku̍kṣi̍ṁ tataḥ svaccha̍to̍yaṁ pibecca\,~॥29॥

\newpage 

pra̍ṇa̍ma̍ prāṇaṁ pra̍tha̍ma̍ṁ yoge\\
bha̍ja̍re̍ prāṇaṁ bha̍ktyā parayā\,~।\\
prā̍ṇāyāmaṁ ku॒ru̍ tatpaścāt\\
dhyā̍tvā praṇavaṁ pa॒re̍śa sadanam\,~॥30॥

mā̍kuru॒ mā̍kuru॒ yo̍gatyāgaṁ\\
mā̍ mā bhakṣaya tā̍masamannam\,~।\\
prā̍ṇaṁ bandhaya ni॒ya̍mānnityaṁ\\
bha̍ja bhaja bhagavatpā̍dadvandvam\,~॥31॥

ba̍ndhaya vāyuṁ na̍ndaya jīvaṁ\\
dhā̍raya cittaṁ da॒ha̍re parame\,~।\\
i̍ti̍ ti̍rumala kṛṣṇo yogī\\
pra॒di̍śa̍ti vācaṁ sa̍ndeśākhyām\,~॥32॥


\end{document}