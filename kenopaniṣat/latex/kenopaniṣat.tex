%!TEX encoding = UTF-8 Unicode
%!TEX TS-program = xelatex

%\documentclass[11pt, a4paper]{article}
\documentclass[parskip, DIV=14]{scrartcl}
\usepackage{fontspec}
\usepackage{xltxtra} % a few fixes and extras
\usepackage{fancyhdr} % header

\usepackage{xcolor} % writing in color

\renewcommand{\baselinestretch}{1.2}% interligne + 20%

%\setmainfont[Ligatures=TeX]{Linux Libertine}
%\setmainfont[Ligatures=TeX]{Sanskrit 2003}
%\setmainfont[Ligatures=TeX]{siddhanta}
\setmainfont[Ligatures=TeX]{Chandas}

% footnote in end of document
\interfootnotelinepenalty=10000

\renewcommand{\thefootnote}{\fnsymbol{footnote}} %A sequence of nine symbols, try it and see!


% add a small title: name of the file + initials
\pagestyle{fancy}
\lhead{\color{lightgray} kenopaniṣat}
\rhead{\color{lightgray} LSN}

\setkomafont{pagenumber}{\normalfont\bfseries}
\pagenumbering{roman}
\begin{document}

%\pagenumbering{gobble} % remove page number

\vspace{-1.5cm}

\begin{center}
% see command for footmark without numbering
\textbf{{\Huge॥\,~kenopaniṣat\,~॥\LARGE\let\thefootnote\relax\footnote{\color{lightgray} Document written using \XeLaTeX{} and chandas font,  version 11/05/2019, LSN.}}}
\end{center}
\Large

%talavakāra upaniṣat (according to śāṅkarācarya)

{\centering	
%\raggedright

%%%%%%%%%%%%%%%%%% Gḥāṇā %%%%%%%%%%%%%%%%%%%%%
\large
Other names:  talavakāra upaniṣat (according to śāṅkarācarya).
9th chapter of the upaniṣad brāhmaṇa of the talavakāra branch of the sāma veda.

\vspace{0.5cm}

\Large
॥\,~śānti pāṭhaḥ\,~॥\\
\vspace{0.5cm}

om āpyā̍yantu ma̍māṅgā॒ni vākprā॒ṇaśca॒kṣuḥ śro॒trama̍tho balami॒ndriyāṇi̍ ca sa॒rvāṇi̍ sa॒rvaṁ̎ brahmaupa॒niṣa̍daṁ mā'haṁ bra̍hma nirā̍kuryā॒ṁ mā mā bra̍hma nirā̍karo॒danirā̍karaṇamastva॒nirā̍karaṇaṁ me'stu॒ tadā̎tmani nira̍te ya॒ upa̍niṣatsu dha॒rmāste̍ mayi̍ santu॒ te mayi̍ santu~॥\\

\vspace{0.5cm}
          ॥\,~oṁ śānti॒ḥ śānti॒ḥ śānti̍ḥ \,~॥\
\vspace{0.5cm}

॥\,~atha kenopaniṣat~॥\\
\vspace{0.5cm}

oṁ ke॒neṣi̍taṁ patati pre॒ṣita̍ṁ manaḥ kena॒ prāṇa॒ḥ pratha̍maḥ praiti॒ yukta̍ḥ~।
kene̍ṣitāṁ vā॒cami̍māṁ vadanti
    cakṣu̍ḥ śro॒traṁ ka u॒ devo̍ yunakti~॥~1~॥
    
śrotra̍sya śro॒traṁ mana̍so ma॒no yadvā॒co ha vā॒caṁ sa u̍ prā॒ṇasya prā॒ṇaśca॒kṣuṣa̍śca॒kṣura̍timucya dhī॒rāḥ
    pretyā̎smāllo॒kāda॒mṛtā̍ bhavanti~॥~2~॥
    
na ta̍tra ca॒kṣurga̍cchati na॒ vāgga̍cchati no॒ mano॒ %var. mana̍ḥ~।
na vidmo̍ na॒ vijā̍nīmo yathai॒tada̍nuśiṣyāda॒nyade̍va tadvi॒ditā̍datho a॒viditā̍dadhi~।
iti̍ śuśruma॒ pūrve̎ṣā॒ṁ ye na॒stadhvyā̍cacakṣi॒re~॥~3~॥

yadvā̎cā'nabhyu॒dita॒ṁ yena̍ vāga॒bhyu̍dyate̎~।
tade̍va brahma tvaṁ viddhi̍ ne॒daṁ ya॒dida॒mupā̍sate~॥~4~॥

yanmana̍sā na ma॒nute̍ yenā̍''hurmano॒ matam̎~।
tade̍va brahma tvaṁ viddhi̍ ne॒daṁ ya॒dida॒mupā̍sate~॥~5~॥

yaccakṣu̍ṣā na pa॒śyati̍ yena॒ cakṣūgṁ̍ṣi paśyati~।
tade̍va brahma tvaṁ viddhi̍ ne॒daṁ ya॒dida॒mupā̍sate~॥~6~॥

yacchro̍treṇa na śṛ॒ṇoti̍ yena॒ śrotra̍midaṁ śrutam~।
tade̍va brahma tvaṁ viddhi̍ ne॒daṁ ya॒dida॒mupā̍sate~॥~7~॥

yatprā̍ṇena na prā॒ṇiti̍ yena॒ prāṇa̍ḥ praṇīyate~।
tade̍va brahma tvaṁ viddhi̍ ne॒daṁ ya॒dida॒mupā̍sate~॥~8~॥
 
\vspace{0.5cm}
 ॥~iti kenopaniṣadi prathamaḥ khaṇḍaḥ~॥
 %kenopaniṣadi locatif 
\vspace{0.5cm}

ya॒di ma̍nyase su॒vede̍ti dabhrame॒vāpi̍ %## var ## daharamevāpi
    nūnaṁ tvaṁ ve॒ttha bra̍hmaṇo rūpam~। 
yada̍sya tvaṁ yada̍sya de॒veṣva̍tha nu mī॒māggsya̍meva te ma॒nye vi̍ditam~॥~1~॥

% mī॒māggsya̍meva or mī॒māṁsya̍meva
    
nā॒haṁ ma̍nye su॒vede̍ti no na॒ vedeti॒ veda̍ ca~।
yo na̍stadveda॒ tadveda॒ no na॒ vedeti॒ veda̍ ca~॥~2~॥

ya॒syāma̍taṁ ta॒sya ma̍taṁ mataṁ ya॒sya na॒ veda̍ saḥ~।
avi̍jñātaṁ vijā॒natāṁ vijñā̍tamavi॒jāna̍tām~॥~3~॥

pra॒tibo̍dhavidi̍taṁ mata॒mamṛ̍tatvaṁ hi॒ vinda̍te~।
ā॒tmanā॒ vinda̍te vī॒rya॒ṁ vi॒dyayā̍ vindate॒'mṛ̍tam~॥~4~॥

iha̍ ce॒dave̍dīdatha sa॒tyama̍sti na ce॒dihāve̍dīnma॒hatī॒ vina̍ṣṭiḥ~।
bhū॒teṣu̍ bhū॒teṣu̍ vici̍tya dhī॒rāḥ pretyā̎smāllo॒kāda॒mṛtā̍ bhavanti ~॥~5~॥

\vspace{0.5cm}    
॥~iti kenopaniṣadi dvitiyaḥ khaṇḍaḥ ~॥
\vspace{0.5cm}       
      
bra॒hma ha॒ deve̎bhyo viji̍gye ta॒sya ha॒ brahma̍ṇo vi॒jaye॒ devā amahī̍yanta~॥~1~॥

ta aikṣantāsmā̎kamevā॒yaṁ vijayo'smā̎kamevā॒yaṁ mahi̍meti~।

ta॒ddhaiṣā̎ṁ vija̍jñau te॒bhyo ha॒ prādu̍rbabhūva ta॒nna vya̍jānata ki॒mida॒ṁ yakṣa̍miti~॥~2~॥

te॒'gnima̍bruvan jā॒tave̍da e̍tadvi॒jānī॒hi ki॒meta॒dhyakṣa̍miti tathe॒ti ~॥~3~॥

tada̍bhyadrava॒ttama̍bhyavada॒tko'sī̎tyagni॒rvā a॒hama̍smītyabra॒vījjā॒tave̍dā vā a॒hama̍smīti~॥~ 4~॥

tasmi̍gṁ stvayi̍ kiṁ vī॒ryami̍tya॒pīdagṁ sarvaṁ̎ dahe॒yaṁ yadi̍daṁ pṛthi॒vyāmiti̍~॥~5~॥

tasmai̍ tṛṇaṁ ni॒dadhā̍vetadda॒heti̍ ta॒dupa̍preyāya sa॒rvaja̍vena ta॒nna śa̍śāka da॒gdhuṁ sa tata̍ eva ni॒vavṛ̍te naita̍daśakaṁ vi॒jñātu̍ṁ yadeta॒dyakṣa̍miti~॥~6~॥

atha vā॒yuma̍bruvanvā॒yave̍tadvi॒jānī॒hi ki॒meta॒dyakṣa̍miti tathe॒ti~॥~7~॥

tada̍bhyadrava॒ttama̍bhyavada॒tko'sī̎ti vāyu॒rvā a॒hama̍smītyabra॒vīnmā॒tari̍śvā vā a॒hama̍smī॒ti~॥~8~॥

tasmi̍gṁ stvayi̍ kiṁ vī॒ryami̍tya॒pīdaṁ sarva̍mādadī॒yaṁ yadi̍daṁ pṛthi॒vyāmiti̍~॥~9~॥

tasmai̍ tṛṇaṁ ni॒dadhā̍vetadā॒datsveti̍ ta॒dupa̍preyāya sa॒rvaja̍vena ta॒nna śa̍śākādā॒tuṁ sa tata̍ eva ni॒vavṛ̍te naita̍daśakaṁ vi॒jñātu̍ṁ yadeta॒dyakṣa̍miti~॥~10~॥

a॒thendrama̍bruvanma॒gha̍vanne॒tadvi॒jānī॒hi ki॒meta॒dyakṣa̍miti tathe॒ti tada̍bhyadrava॒ttasmā̎ttiroda॒dhe~॥~11~॥

sa tasmi̍nnevākā॒śe str॒iyamā̍jagāma ba॒huśo̍bhamānāmu॒mā॒gṁ haima̍vatīṁ tā॒gṁ ho̍vāca ki॒meta॒dyakṣa̍miti~॥~12~॥

\vspace{0.5cm}    
॥~iti kenopaniṣadi tṛtiyaḥ khaṇḍaḥ ~॥
\vspace{0.5cm}    

sā brahme॒ti ho̍vāca brahmaṇo॒ vā e̍tadvi॒jaye॒ mahī̍yadhvamiti tato̍ haiva vi॒dāñca̍kāra brahme॒ti~॥~1~॥

tasmā॒dvā e॒te de॒vā atita̍rāmi॒vānyā॒ndevā॒nyada̍gnirvā॒yuri̍ndraste hyena̍nnedi॒ṣṭhaṁ paspa̍rśu॒ste hyena̍tpratha॒mo vi॒dāñca̍kāra brahme॒ti~॥~2~॥

tasmā॒dvā i॒ndro'tita̍rāmi॒vānyā॒ndevā॒n sa hyena̍nnedi॒ṣṭhaṁ paspa̍rśa sa॒ hyena̍tpratha॒mo vi॒dāñca̍kāra brahme॒ti~॥~3~॥

ta॒syaiṣa āde॒śo yade̍tadvi॒dyuto॒ vyadyu̍tadā3 itīnnya̍mīmiṣadā3 itya̍dhidai॒vatam~॥~4~॥
% ##Extra `ā'kār is used in the sense of comparison##

% dā6 = 6 temps (Radha)

athādhyā॒tmaṁ ya॒ddeta॒dgaccha̍tīva॒ ca॒ mano'ne̍na caita̍dupasmara॒tyabhī̎kṣṇagṁ saṅka॒lpaḥ~॥~5~॥

taddha॒ tadvanaṁ̎ nāma tadva̍namityu॒pāsi̍tavya॒ṁ sa॒ ya॒ eta̍devaṁ ve॒dābhi̍haina॒gṁ॒ sarvā̍ṇi bhū॒tāni̍ saṁvāñcha॒nti~॥~6~॥

upaniṣadaṁ̎ bho brū॒hītyu̍ktā ta upa̍niṣad brā॒hmīṁ vā॒va ta॒ upa̍niṣadamabrūme॒ti~॥~7~॥

ta॒syai tapo॒ dama॒ḥ karme̍ti prati॒ṣṭhā ve॒dāḥ sarvā̎ṅgāni satyamā॒yata̍nam~॥~8~॥

yo̍ vā e॒tāme॒vaṁ ve॒dāpa̍hatya pāpmā̍namanante sva॒rge lo॒ke jye̍ye prati̍tiṣṭhati॒ prati̍tiṣṭhati ~॥~9~॥

\vspace{0.5cm}    
॥~iti kenopaniṣadi caturthaḥ khaṇḍaḥ ~॥
\vspace{0.5cm}    

\vspace{0.5cm}
          ॥\,~oṁ śā॒ntiḥ śā॒ntiḥ śānti̍ḥ \,~॥\
\vspace{0.5cm}

॥\,~iti kenopaniṣat \,~॥\\

%%%%%%%%%%%%%%%%%%%%%%%%%%%%%%%%%%%%%%%%%%


\end{document}